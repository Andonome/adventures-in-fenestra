% !TeX root = ../main.tex
We've looked around the edges of Fenestra rather a lot, with charts, and races, and hints in the corebook.  Let's get stuck into how it looks on the ground.

\section{Monsters, Malthus \& Diseases}

Left to their own devices, humans populations will explode until they run against the buffer of limited resources.  There are three ways for a human population to limit their numbers -- war, disease and famine.  When famines strike, many find war preferable to starving to death; one has the opportunity to take another's fertile land.  A reason to go to war is always an afterthought to the notion of second sons of the family -- those who will not inherit the means of food production, and therefore will not be able to provide for themselves or create a family.

Disease also does its share to limit population growth, but within Fenestra, diseases are rare.  No smallpox nor any plagues ravage the land.  However, they have to put up with a new type of population limitation: monsters.  Humanoids are the natural prey of basilisks, dragons, griffins, woodspies and many more creatures.  While in our own history, it was natural for populations to grow and then wane as a new wave of diseases hit cities, the cities in Fenestra are more likely to suffer from starving lower classes rather than see a wave of plague.  The villages are where one can freely grow food.

\subsection{Villages}

Due to the monstrous populations, many villages have become walled.  However, there is far less danger within villages than outside.  If a wandering basilisk finds a human village, it is far more likely to try to eat the cows than the people.  The villagers' large numbers also provide them a good opportunity to drive beasts away with arrows or even by simply throwing rocks.

The roads between villages poses far more danger, so the most dangerous job of the land is always that of the independent trader or message carryers.  Few people dare to travel alone throughout the countryside.
Forests and plains alike hold terrifying beasts.  Fast horses are highly prized as a means of safety for couriers.  Others remain safe by travelling in large caravans, often with a sacrificial cow -- a beast or two which can be given to any dangerous creatures or bandits who attack.  Braver groups will travel with bows.

Humans of Fenestra tend to view nature as an enemy encroaching upon the little segment of the world they can make their own, and quite understandably so; when creatures attack villages so often, they sometimes succeed too well, and entire villages can be pulled `from the map', as everyone inside dies or flees.

\subsection{Forests}

Near villages, people use forests to grow fruits or cultivate trees for particular wooden tools.  These `tame' forests are generally safe.

Beyond the safe borders, the forests become wild.  People forced outside of society by the threat of starvation often form bandit groups out of sheer desperation.  These groups never last long due to sleeping beside so many dangerous creatures, but the forests' unspoilt nature means that food is often plentiful to those with the skills to forage for it, so the bountiful trees always beckon to the hungry.

Not long after, wild forests become uninhabitable.  Chitincralwers, woodpies, and worse lie in wait to eat anyone nearby.  In these deep regions, gnomes and elves survive by living underground, but can never wander the world with any real safety.

\section{Regions}\index{Encounters}\label{encounters}

Fenestra contains six regions, each of which have a slightly different ecology and different populations.

The human-populated areas tend to defend themselves and drive out dangers and monsters, but wander too far between settlements, or make a journey into a forest, and you may find a nasty creature, random traders, or even a mana lake.\footnote{See page \pageref{mana_lake}.}

Each region has a set of standard encounters for times when players wander into less populated and more dangerous areas.  These encounters aren't necessarily there for combat.  If players spot wolves, the pack may simply stalk them with a mind to steal food.  Alternatively, the players may wander into a pack of wolves mid-hunt, bringing down a deer.  An encounter with griffins need not be violent -- they could simply see a nest in the distance, and make the decision to steer clear.

Encounters can flavour a journey in a multitude of ways, and allow those characters with Beast Ken or the Aldaron magic sphere to interact with the world around them.

\subsection{Bearded Mountains}\index{Bearded Mountains}

The Bearded Mountains themselves are riddled with dwarvish tunnels and occasional little gnomish communities closer to the surface.  It is said that you can travel from anywhere to anywhere in this region via underground passages, if you have a gnomish guide.  Travelling without such a guide is, of course, likely to ensure a nasty encounter with an umber hulk or acidic jelly.

Farther from the mountains and closer to the sea, seafaring humans live and trade with the Pebbles islands.

The bearded mountains are famed for their terrible storms.  During the stormy seasons -- Qualmea and Otsea -- fishermen do not go out to sea.  Tidal waves explode across the oceans.  Tsunami are common.  Worst of all, the Bearded Mountains themselves often explode, sending lava flowing downhill and great clouds of smoke and ash into the air.

The dwarves extensively study which tunnels fill with magma, and eagerly await dormant tunnels where they can build new homes where the magma has subsided.  Humans in the area often herd their animals into barns or their own houses, and then put themselves into small underground bunkers to await the passing of the storm.

\subsection{Underground}
The tunnels winding their way around the deeps can provide a relatively safe passage from A to B, though one must know which road to take.  This maze of underground forks connects gnomish warrens and dwarvish fortresses.

\label{bearded_encounters}

\begin{encounters}{the Bearded Mountains}
	Deep Roads & Plains & Result \\\hline
	\li &  Mana Lake (page \pageref{mana_lake}). \\
	\li &  Umber Hulk (page \pageref{umber_hulk}). \\
	\li &  A band of dwarvish bandits. \\
	\li &  Acidic jelly (page \pageref{jelly}). \\
	\li \lii  A band of 2D6 gnomish traders(page \pageref{gnomish_citizen}).  \\
	\li \lii  A band of dwarvish traders (page \pageref{dwarven_trader}). \\
	& \lii  A band of human bandits. (page \pageref{human_soldier})\\
	& \lii  Griffins. (page \pageref{griffin})\\
	& \lii  A bear. (page \pageref{bear})\\
	& \lii  A band of human traders (page \pageref{human_trader}). \\
\end{encounters}

\subsection{Dogland}\index{Dogland}

A generation ago, Dogland was a place with sparce human villages, but no large settlements.  Elves, gnomes and (mostly) gnolls inhabited the area.  The fighting broke out shortly afterwards, and human won the war, then enslaved the gnolls The fighting broke out shortly afterwards, and human won the war, then enslaved the gnolls.

Since then the Gnolls Guild has formed, and worked tirelessly to build new settlements\ldots or at least worked the gnolls tirelessly.  The region has since provided excellent logging, and the nearby forests have been tamed fairly quickly.  However, the deep forests are still fearful places to tread.

Upon the coast, a second branch of the College of Alchemy has opened, devoted entirely to martial magic.  The King has allowed the wizards to operate fairly independently, and with a lot of funding, in order to gather information on any potential attacks upon the Pebbles islands, nearby.


\begin{encounters}{Dogland}
Deep Forest & Open Road & Result \\\hline
	\li & Sleeping dragon (page \pageref{dragon}). \\
	\li & Mana Lake (page \pageref{mana_lake}). \\ 
	\li & A rogue band of gnoll raiders (page \pageref{gnoll_hunter}). \\ 
	\li & Chitincrawler (page \pageref{chitincrawler}). \\ 
	\li & Woodspy (page \pageref{woodspy}). \\ 
	\li & Swamp. \\ 
	\li \lii Mouthdigger (page \pageref{mouthdigger}). \\ 
	\li \lii Bear (page \pageref{bear}).\\
	\li \lii Wolves (page \pageref{wolf}).  \\
	& \lii Human Traders (page \pageref{human_trader}). \\
\end{encounters}

\subsection{Eastlake}\index{Eastlake}

	Soggy, miserable children -- mostly with rich parents -- are forced in their hundreds to the College of Alchemy to receive training in history, invocation, literature and most popular of all -- conjuration.  Most are from warmer lands, and have a hard time dealing with the cold.  Mid-way up Eastlake, overlooking the great lake which the area is named for, the College watches over many leagues.  At night, it is possible to see little hearthfires from the windows of village cottages.  Evergreen trees dot the landscape and then turn into thick forests further North.

	King Wyatt, worried about the possibility of creating a ruling class of powerful alchemists, has banned all alchemists from owning land.  For this reason, first sons are almost never sent to study in the college.  After their studies, many return to their families with a few magical tricks but often lead solitary lives as they can neither own their own land nor till another's -- such activity would be beneath the nobility.  Some few become village mages, neither owning land nor working in the normal way but rather gaining a perpetual stream of revenue from a nearby village for entertaining them, fixing problems and 

	The official god of the region was Qualme, but after the College was erected C\'{a}le became the favourite as so many followed the examples of the mages in the area.  Massive halls full of the writings of C\'{a}le fill the college like miniature chapels full of reading rather than pews.

	Still farther North, snow-elves live in icy caverns or build castles made partly of ice and partly of enchanted evergreen conifers.  They hunt with a combination of spears and enchantments and occasionally battle with gnoll incursions from Whiteland.

	A little farther North still, a handful of those forgotten temples to Qualme remain with animate, but not living priests, who decided never to die, but to study and pray forever.  While many rest meditatively, enough have become resentful that the region now has a real problem with roaming undead.

	An empty wasteland sits between the elvish forests and the dwarvish mountains to the South in the Bearded Mountains region.  Historically, little wars have opened up when dwarves came North to chop down the massive trees for wood.  The elves guard their forests fiercely, partly because fewer trees means fewer animals to hunt and partly because they feel a deep connection with the area.

	\subsection{The College of Alchemy}
	The four houses -- Alisa, Kisha, Stein and Ventress -- run the college as a union, and have successfully managed to strangle a lot of the realm's alchemical potential.  Deals with the Crown have been legislated, and mages may no longer swap magic of any kind -- all magical learning must pass through the guild and pay for the privilege.  Any magic practiced or learned outsideof the Guild's Monopoly is classified as `black magic', whether this is nuramancy or simply unregistered alchemy.


\begin{encounters}{Eastlake}
	Snowy Wastes & Lakeside & Result \\\hline
	\li & A discarded magical item, left in the wastes, perhaps next to some wizard's body. \\
	\li & A contingent of ghouls (page \pageref{ghoul}). \\
\li \lii An undead knight (page \pageref{ghast}). \\
	\li \lii Wolves (page \pageref{wolf}). \\
	& \lii A bear (page \pageref{bear}). \\
	& \lii Aurochs (page \pageref{auroch}). \\
	& \lii Human Traders. \\

\end{encounters}

\subsection{Mt Arthur}\index{Mt Arthur}

Mt Arthur is a thriving region full of towns and villages full of bountiful crops.  Cities in the area host gladiator matches where people can be legally bought and sold for a limited period of time as a sort of indentured servitude.  This is a common means of escaping the noose as criminals are often permitted to fight to the death rather than hang.

	This area has seen much warfare with the South Kingdom, but in recent years, as the memories of war fade, old Paths between the kingdoms are once again being trodden, but this time by intrepid traders.  There is much profit to be made going between the two areas, as each has items considered rare to the other.  While the South is richer in gems and fine silks, the North has more metals and meat.

	Dwarven settlements are dotted about the mountains, acting as a peaceful area for both sides to negotiate and trade, and often imposing little taxes and tribute demands for the freedom to do so.

	Far from the dwarves, other areas of the mountains are populated by little elvish communities near the surface, and by many a dwarvish community below.  The elves tend to operate near the surface and almost always have to find someone else to do their tunnelling for them, while the dwarves build far more elaborate tunnels, often going far to deep to be safe from the things which live below the earth.

	V\'{e}re is by far the most popular god within the region, and temples dot the land to him, producing contracts for marriage, business deals and often acting as court-houses.

	Mt Arthur can suffer storms almost as bad as the Bearded Mountains region.  During the stormy seasons lightning and massive amounts of hail are common.  The Kingsway mountains at the South end of Mr Arthur, dividing the two kingdoms, are rife with volcanic activity, but most of the region is far enough away from the mountains to avoid any problems.  As a result the stormy seasons can often be cause for celebration as people relish the awesome sight of a releasing volcano.



\begin{encounters}{Mt Arthur}
	Deep Forest & Open Roads & Result \\\hline
	\li & Mana Lake (page \pageref{mana_lake}. \\
	\li & Bassiliski (page \pageref{basilisk}). \\
	\li \lii Band of travelling elves (page \pageref{elf}). \\
	\li \lii Chitincrawler (page \pageref{chitincrawler}). \\
	\li \lii Wolves (page \pageref{wolf}). \\
	\li \lii Bandits (page \pageref{human_soldier}). \\
	& \lii Human Traders. \\
	& \lii Travelling pilgrims. \\
\end{encounters}

\subsection{The Pebbles}\index{Pebbles}

These little islands had their own group of languages -- very different from any of the various human languages around the mainland of Fenestra.  While they traded with both Fenestra and the South Kingdom, they were until recent decades, independent.  They had no lords or armies, but did rally around priests of Ohta in times of war.

	Since the Pebbles' annexation with Fenestra, they have had nothing but trouble as their ports were used in wars with the South Kingdom.

	They commonly worship Alasse and have some of the grandest temples build in her honour, full to the brim with beer and joyful songs.  Many a village here hosts a single `village gnome' -- typically an alchemist who teaches the young people history and sometimes even a little magic.  Local gnomish communities consider it good practice to create ties to humans, though they do not take anyone back to their own homes, claiming gnomish warrens would be `too short by half' to humans to visit.

	King Wyatt's personal wing of the Shattered Castle sits in the main island of the Pebbles.  It has a portal in the roof which looks out sideways on a waterfall.  While the spectacle appears amazing, many an onlooker has felt queasy looking up to see a waterfall going in the wrong direction.


\begin{encounters}{the Pebbles}
Open Sea & Land & Result \\\hline
\li & Raging Storm. \\
\li & Pirates. \\
\li & Mild Storm. \\
& \lii Gnomish Illusionist. \\
& \lii Mouthdigger. \\
\end{encounters}


\subsection{Quennome}\index{Quennome}

Throughout the Quennome forests, connected trees fashioned into houses form subtle living spaces.  It can often be impossible to tell what counts as a `tree village' and what counts as a `tree castle'.  Under the ground, little elvish communities spring up here and there.  Mostly, the forest's base is reserved for creatures rather than people, and it is easy to see why after one sees the kinds of creatures which roam there.  Basilisks, griffins, woodspies and sometimes nura wander the landscape, looking for food.

	Nestled within the Quennome forest are little human towns.  Perhaps in imitation of the elves they build their houses slightly below the ground so that a thatched roof on top of two feet of brick wall is all that can be seen.  The people are adept at rallying round and defending their villages from attacks by larger beasts.  Archery with the long bow is popular for just this purpose and Quennome boasts the best archers in Fenestra.

	The monstrous beasts in the area often like to capture humans when they are walking in smaller numbers, so they sit in waiting around human Paths.  As a result, people change how they get from one place to the other very often.  Feint Paths appear, are forgotten and quickly vanish.

	Various temples to Laique are erected with stone or wood, and statues created in honour of all the most terrifying beasts of the forest.  It is hoped that if they are treated with respect then they will leave people alone.



\begin{encounters}{Quennome}
	Deep Forest & Lakeside Areas & Result \\\hline
	\li & Elvish Enchanter (page \pageref{elven_enchanter}). \\
	\li & Mana Lake (page \pageref{mana_lake}). \\
	\li & Dryad (page \pageref{dryad}). \\
	\li \lii Bassilisk (page \pageref{basilisk}). \\
	\li \lii Chitincrawler (page \pageref{chitincrawler}). \\
	\li \lii Woodspy (page \pageref{woodspy}). \\
	\li \lii Travelling elves (page \pageref{elf}). \\
	\li \lii Aurochs (\pageref{auroch}). \\
	\li \lii Griffins (page \pageref{griffin}). \\
	\li \lii Bear (page \pageref{griffin}). \\
	\li \lii Wolves (page \pageref{wolf}). \\
	& \lii Heavily armed band of Human Traders. \\
\end{encounters}

\subsection{Whiteland}\index{Whiteland}

Small villages and rare towns dot this snowy expanse.  The land has grown unstable in recent years as the crown has killed all nobles in the area and left the people with little official bureaucrats who are in charge of collecting taxes and meting out punishments.

	In a desolate and forgotten region of Whiteland is a massive structure without any footPaths leading to it or from it.  It is built in a dome shape so that snow covers as much as possible.  It has no doors or windows on the outside except one -- a single door at the top goes to a walkway so guards can see attacks coming from the distance.

	Inside this monolithic white dome wind stone corridors.  Some have doors, and some doors have locks, but it seems like there is no plan or pattern to any of it.  Corridors vary in size from some barely large enough for a man to squeeze through while crouching, and others large enough to get a full wagon through.  Various hallways exit to magical portals, created through alchemy.  One exits to a wing in Mt Arthur, another goes out to Quennome.  All in all each region has one portal which leads to the Heart of the Shattered Castle, and from the heart one can travel to any other.  And behind one, lonesome and secret, locked and guarded door, sits the portal to the Pebbles Wing where King Wyatt currently resides.

	Various traders pay a steep fee to travel from Wings of the Shattered Castle in each region to the Heart, and then out into another region.  Some of these portals activate with passwords but most are permanently simply turned on.  All command wordsare a jealously guarded secret, though it is rumoured that the College of Alchemy has been gathering them.

	There are other rumours concerning the heart, of secret rooms with locked doors, with gates which lead to other worlds.  Perhaps these are places so far from Fenestra that they appear strange.  Some think that these doorways lead to Ainumar -- the great celestial orb in the sky where it is said that the gods live.

	Far in the North, where the snow never melts, caves of gnolls fight and occasionally journey South to raid human villages.  They have been quieter in the years since the great war in Whiteland, when the gnolls of the North coordinated their attacks with those of Dogtown.



\begin{encounters}{Whiteland}
South & North & Result \\\hline

\li &  A hidden portal to one of the etherial lands. \\
\li & Mana Lake. \\
\li \lii Gnoll hunters. \\
\li \lii Snowstorm. \\
\li \lii Wolves. \\
& \lii Soliders in training \\
& \lii A hidden human settlement that doesn't like paying taxes. \\
& \lii Human Traders. \\
\end{encounters}

