\chapter{Village Encounters}

\setcounter{encnum}{1}\renewcommand{\encsymbol}{\ding{170}}

\section{Summaries}

\begin{multicols}{2}

\sidequest{Here and No Farther}

\gls{baron} wants to expand his domain into a nearby forest, but \gls{forestpriest} has other plans.
Page \pageref{herenofarther}.

\begin{enumerate}

	\item{\glsentrytext{forestpriest} found a man threatening to rebuild the lost city in the forest, so he turned that man into a deer. This leaves the party with a confused deer.}
	\item{(Forest) The party stumble across a mason garrison, starting to invade the forest.}
	\item{(Town) \Glsentrytext{forestpriest} found another man endangering the forest, so he was turned into a sheep in town.}
	\item{(Town) \glsentrytext{forestpriest} is labelled a criminal and flees.}

\end{enumerate}

\sidequest{The Necromancer's Pets}
A necromancer slowly builds an army of the dead.
Page \pageref{necromancerspet}.

\begin{enumerate}
	\item{A crow raises a body from the dead.}
	\item{(Forest) The characters are attacked by an undead horde.}
	\item{A band of traders have been killed, but followers of Qualm\"{e} were left alive.}
	\item{Night of the dead.}
	\item{\ding{106} The necromancer assaults a village.}
\end{enumerate}

\sidequest{Immortal Bandits}

The ``Immortal Bandits'', are lead by \gls{banditking}.
They carry magical rings which bring them into a state of semi-undead, and live among the dead in a temple owned by a necromancer.
Their horses are also undead, which allows them to travel farther than any natural army, with less need for rations or rest.
So far nobody has been able to track them down, but even if they did, their fortress is unassailable, except by a full army.

\Gls{banditking} raids the nearby villages for food to bring to his brother, \gls{sewerking}, and for his personal wealth.

Page \pageref{immortalbandits}.

\begin{enumerate}

	\item{Bandits hide at the side of the road, but don't trouble the characters.}
	\item{A local village has had all its feast supplies stolen by bandits, and the grave of a dead hero raided.}
	\item{A local tracker returns as a ghoul, wearing a magical necklace.}
	\item{A wagon of traders are found stuffed with arrows.}
\end{enumerate}

\sidequest{Desperate Measures}

A local Villagemaster and his family have been transformed into nura monsters, and they are eating enough to starve out the local village.
Page \pageref{desperatemeasures}.

\begin{enumerate}

	\item{\ding{106} Pathetic bandits assault the characters.  They were villagers but have turned to banditry out of hunger.}
	\item{A wagon with chickens is journeying \emph{from} town.}
	\item{\ding{106} (Town) A diplomat requests the characters track down a priest to cure a local lord of a curse.}
	\item{\ding{106} (Town) A town crier says that a bounty has been called on a local lord.}

\end{enumerate}

%\subsection[The Message]{The Message -- \encnum}
%\gls{nurabaron} has made a deal with a mysterious figure from deep underground.  He wants to raise and army and dispossess the crown.  He has many friends already, but knows he needs to gain assassins and hidden powers if he wants to play this right, so he's bargained for magical items.

%The deal so far is clear -- a limited shipment of food is to be delivered.  Some of it has been delivered already, and the rest is calculated in a book \gls{nurabaron} keeps hidden in his room.
%
%Page \pageref{message}.
%
%
%\begin{enumerate}
%
%\item{The bandits robbed the courier blind, but his message remains in his hand.} % The message is about receiving a magical item.
%\item{Someone murdered an entire household, and monsters sit in the atic.}
%\item{\ding{106}\ding{106}  Villagers have turned into nura monstrocities.}
%\item{A nobleman insults and taxes the characters.}
%\item{\ding{106}\ding{106}  Bandits eat a trader's corpse -- all have turned into nura.}
%
%\end{enumerate}

\setcounter{encnum}{1}

\end{multicols}

\section{Encounters}\index{Encounters!Villages}

\begin{multicols}{2}

\subsection[Here and No Farther]{Here and No Farther -- \encnum}\label{herenofarther}

Centuries ago, men created a logging city, and the elves destroyed it.  Recently, \gls{baron} has decided to reclaim it, though nobody knows exactly where it is.  He sent a dozen trackers to find the place in the hopes of rebuilding it; but \gls{forestpriest}, doesn't like the situation, and has been turning anyone who finds or hunts for the place, into animals.

\villages{You See a Deer}

\begin{boxtext}

	A startled deer stops in front of you, then just stops stares.

\end{boxtext}

Yesterday, Jade the tracker came to \gls{lostcity}, but \gls{forestpriest} didn't like the intrusion, so he turned the intruder into a deer.
Jade fled instantly.

If the characters begin to question the odd behaviour of the deer, they can make a Wits + Beast Ken roll to notice that something is wrong, TN 10.

Anyone attempting to dispel the magic must roll at TN 12, so in all likelihood the deer will remain a deer.

\NPC{\M}{Jade the Tracker}{Personable}{Plays with knife}
\person{1}{1}{0}{{0}{0}{1}}{0}{1}{Beast Ken 2, Empathy 1, Survival 2}{Nothing, not even clothing.}{}

If returned to human form, Jade can instantly identify \gls{forestpriest} as the man who turned him, though in deer form he will have a lot of trouble expressing himself, and can do little more than show distress around the miracle worker.

The characters may decide to visit \gls{forestpriest} in order to have the spell lifted, but \gls{forestpriest} simply tells them that this is an ordinary deer, enchanted to think that it is a man.

Move this encounter to the Forest.

\forest{The Green Tower}

Deep in the forest, \gls{baron} has commissioned masons to start building a single tower from the fallen stone of \gls{lostcity}. This is the first forway into the deep wilderness.  The top is to be disguised by painting is green (the color is made by malachite, which can be expensive).

\begin{boxtext}

	In the deep forest, where there should really be nothing, you see a little tower, and a man hanging out the window, painting it green.
	With another day of painting, the green tower could have beem made almost invisible from a distance.

\end{boxtext}

The players may watch the building work and report back, or confront the builders, or any number of things.
The builders have no idea that directly under their feet, a forgotten portal to the nura realm sits semi-open.
It's unlikely, at this stage, that they find the secret portal to catacombs of the old Temple of Qualm\"{e}.

\Gls{nightguard} leads the group.
He knows he's meant to keep the place secret, but the masons have no plans to harm the characters unless they are attacked.

See page \pageref{green_tower} for more details.  Move this encounter to the town.

\town{Rogue Sheep}

Run this encounter in Town, at the same time as the encounter below it.

\begin{boxtext}
	A man in ragged clothing, missing a couple of fingers and a couple of teeth, chases a sheep down the road.  A woman shouts out \emph{``Oi, Trevor! That ain't your sheep!''}.

	\emph{``Well whose bloody is it then, bitch?!''}, the ragged man snarls back.

	The sheep runs into a tavern, and Trevor runs in after it.
\end{boxtext}

Yesterday, \gls{baron} and a number of his men met to discuss further building inside the deep forest, and hunting for \gls{lostcity}.
One of the maids who worships at the temple of Laiqu\"{e} on the Town's outskirts heard this and told \gls{forestpriest}.
\Gls{forestpriest} tracked down the main architect, Darren, cornered him in an empty ally, and turned him into a sheep with his polymorphing ability.

The sheep goes into the \gls{pig} because it's the only place of safety he knows.  The sheep is obviously sentient, if anyone bothers to ask him clear questions.

Before the characters can decide what they're doing, start the next town encounter.

\town{\glsentrytext{forestpriest} Exiled}

\Gls{forestpriest} has been found to be a criminal after \gls{captain} stalked him, and found him turning a mason into a pig.
The pig ran away, and was murdered and eaten before \gls{captain} could find him, so `cannibalism', has been added to \gls{forestpriest}'s list of charges.
\Gls{forestpriest} has since fled to the deep forest to escape trial.

\begin{speechtext}

	Hear, ye! Hear, ye!

	A criminal from the Whiteland former nobility has been found, and shall be executed tomorrow, noon, together with three thieves and one man who blasphemed the crown.

	\Gls{forestpriest}, former high priest of Laiqu\"{e}, has been caught using black arts, including cannibalism, and now carries a total reward of 200 silver pieces living, and 300 silver pieces dead.

	All elves in town must carry with them a letter of registration, stating their business in town, and their current lodgings.  Reporting an unregistered elf brings a 50cp reward.

\end{speechtext}

Elvish characters may have a difficult time with the registrations, since everyone in town will ask them for documentation in the hopes of turning them in for a reward.

The criminal of the Whiteland Nobility is someone in hiding, who's come to meet with \gls{banditking} but would rather hang than give up his kin's whereabouts.

\subsection[The Necromancer's Pet]{The Necromancer's Pet}\label{necromancerspet}

\Gls{necromancer} has lived as the lone guardian of his temple for centureies, and has grown increasingly paranoid, so he wants to gather an army of the dead to guarantee his safety.
He started by enchanting a bauble which raises people as ghouls, then enchanted crows to place the medalion on the dead to raise them.
After a century of slowly growing his little dead bettalion, he is ready to assault a full village, and expand it into an army.
Every villager he kills becomes another soldier for his army of the dead.

\villages{The Crow}

Play this encounter at the same time as the one below.  If a corpse appears during the encounter, that's where the crow goes.  If no corpses appear, the characters find a trader by the road, robbed by unknown bandits, and the crow on top of his body.

\begin{boxtext}
	A crow lands upon the corpse and stares at the vacant eyeballs, before landing a sturdy peck on the left one.

	A little glimmer around its neck shines.  The crow wears a medalion, and a moment later it takes off, leaving the right eyeball undigested.

\end{boxtext}

A moment after the crow has flown away, the corpse animates, and attacks the characters.

\npc{\U}{Ghoul}
\ghoul

Even if this ghoul doesn't join the characters, \gls{necromancer} has gathered an army of 50 ghouls by looting the dead.

This encounter moves to the forest area.

\forest{The Undead Hord}

\Gls{necromancer} is not a precise creature, and has misplaced twenty of his ghouls.
Once they wandered away from th rest of the herd, they broke into the forest to wander some more.

\begin{boxtext}

	Crackling sticks indicate someone walks closeby, and a moment later indicates a full procession walking somewhere closeby.
	But you wait, and no voices come out -- only crackling sticks.

\end{boxtext}

The party makes a Wits plus Vigilance roll, TN 6, to notice the undead while they are still 10 squares away.
Each roll on the margin indicates an additional 10 squares to notice the horde, so rolling 10 means 50 squares' distance.

The horde has wandered away from its necromancer, while he was distracted.
There is no way to use them to find where the necromancer lives, as they have no desire to return to him.

This encounter then returns to the Villages.
While these ones have run away, the necromancer has gathered a full army of 100 ghouls.

\villages{The Survivors}

Last night an undead horde pulled apart an entire village.  The undead were directed well by the necromancer, and almost nobody escaped.
However, four people were evident followers of Qualm\"{e}, god of death.
Two wore medalions to commemorate departed loved ones, and another two gave prayer to the God of Death when the undead came.
As a result, they were spared.

Rupert, Jake, Sarah and Eliza cannot explain why they were spared -- they only know the undead never touched them while they ate the rest of the village.

\vfill\null
\begin{boxtext}

	On the horizon, four humanoid sillhouettes stumble forward silently.  Once they see you, they start running towards you.

\end{boxtext}

If the characters track the ghouls, they find it rains en route, making tracking more difficult.
However, a Wits + Survival roll, TN 10, will get them there.
Alternatively, researching previous churches to Qualm\"{e} within the area with an Intelligence + Academics roll (TN 10) will tell the party where they need to go.

By this time the necromancer has gathered an army of 200 undead.

\villages{Knight of the Dead}

The necromancer has gathered an army of a full 400 undead, and has decided to take most of them to a nearby village.  Most are not under his control -- he uses his Necromancy sphere to control some, and the rest usually follow.

\begin{boxtext}
	The village falls quiet at night, except for shuffling feet as people try to be quiet going to the toilet at night, or chattering about local town gossip.

	A man in the distance tells his child off for going out at night into the forest with his friends.
\end{boxtext}

The necromancer has decided it's time to gain more power.  He has the village surrounded.
His tactics are to create a full ring of undead around the village, and pull it slowly tighter until the entire village is surrounded.
Once there is nowhere to go, he releases the undead to attack.

The party can make a Wits + Vigilance roll, TN 10, to spot the dead before they attack.

If they succeed, they can start organizing for the battle.

\paragraph{The Battle} depends upon a single Tactics roll.
If the party see the dead beforehand, they can make an Intelligence + Tactics roll, TN 8.
Otherwise, they can make a Wits + Tactics roll, TN 11.
Success indicates that the dead must back away, while failure indicates that the dead are too numerous, and the party must back away in order to survive.

For tactical purpose, divide the village into four quadrants -- perhaps `the well', `the hallway', `the fields', and `the shrine', or whatever fits the village the characters have ended in.

Each quadrant is attacked by 50 ghouls, while the necromancer stays around the outside, picking off anyone who tries to escape.  He starts with hit hunting bow -- his undead sight allows him to spot people in the dark with ease.  After that, he uses magic, starting with curses, then invocation magic.

The entire village attack together.
While many may attack each round, each character only has to face the number of ghouls attacking them personally.
Use the following as a guide:

\begin{rollchart}

	Round 1 & 3 ghouls attack. \\

	Round 2 & 3 ghouls attack. \\

	Round 3 & Nearby houses go down, torn apart by the dead. \\

	Round 4 & 2 ghouls attack. \\

	Round 5 & All houses a broken, and the dead invade. \\

	Round 6 & Those already dead rise again as the necromancer completes a spell. \\

	Round 7 & 2 ghouls attack. \\

	Round 8 & The dead retreat at the necromancer's order, and any dead villagers return with him. \\

\end{rollchart}

Once the battle's over, the village will most likely be mostly destroyed, and \gls{necromancer}'s army much larger.  The villagers will shake, huddle together, and most consider moving.

The encounter repeats until the party slay \gls{necromancer}.
The characters can complain to local nobles, but they note that they are not allowed a standing army, so they can do nothing about the situation.
The party can write to \gls{king}, but they will be lucky to receive anything but stock letters from beaurocrats stating that the situation has complications.

\npc{\G\U}{200 Ghouls}

\ghoul

\subsection{Immortal Bandits}\label{immortalbandits}

\Gls{banditking} and his men are a bandit ring with a difference.
Most bandits have to raid local areas, until trackers in the Night Guard eventually catch and kill them.
\Gls{banditking} and his team have rings which make them semi-undead, and hide in a lair surrounded by ghouls.
Their undead horses walk slowly, but never tire, so \gls{banditking}'s band can raid villages for many miles around before retiring to their impregnable keep.\footnote{See page \pageref{necromancers_lair} for more on their fetid living conditions.}

The Immortal Bandits have used their position to raid local areas for food and wealth.
All of this is eventually brought back to \gls{sewerking} in order to trade more with the nura.

\villages{Hidden Eyes}

\gls{banditking} and twenty men hide in the dense trees on a hill, a mile away from the road.  They watch, quietly, for traders.  A single man, Engel, sits closer to the road, waiting to call like a pidgeon if someone rich wanders onto the road, or call like a crow if an armoured troop wander along.

\begin{boxtext}

	The road through over to the hamlet is quiet, with only rustling trees, and a crow cawing in the distance.

\end{boxtext}

Nobody will bother the characters, unless they look both rich and unarmed.

Spotting the fake call requires a Wits + Beast Ken roll, TN 10.

If the characters capture Engel, he claims to be a hunter.
If they come for the rest of the bandits, they flee on horseback.

\NPC{\M}{Engel}{Brash}{Licks lips}

\humansoldier

\villages{The Feast is Cancelled}

\begin{boxtext}

	The village looks like some kind of inverted funeral.  A woman stands in a beautiful beige dress at a wooden alter to V\'{e}r\"{e}, consoled by a man wearing the traditional bright wedding hat.  Three dead men lie at the altar's feet, including the a priest.  The village has feasting tables out, but all sit empty.

	A crow attempts to land on one of the dead men, but a local man lunges at it, then gives chase, as if trying to chase the crow into the sky.

\end{boxtext}

Earlier, \gls{banditking} and his men came, took all the food they could, murdered three men who drew swords, then left.

The bandits are ten miles away, and can still be tracked with a Wits + Survival roll at TN 10.
If the PCs catch up to them, the bandits run away.

Characters with a good eye (Wits + Crafts, TN 11), can spot that the crow has a medalion around its neck.

\villages{The Dead Tracker}

Anderson was hired to track down the bandits.
However, the Necromancer's pet crow spotted him, and informed the bandit leader, \gls{banditking}.
He was caught, and killed, then left for dead; but one of the necromancer's crows found the corpse and turned it into a ghoul.
Instead of going to the necromancer, he wandered aimlessly.

The party may notice by his green clothing that he serves in the Night Guard.

\ghoul

If the party enquire with the Night Guard, they will be told where he was wandering to search for \gls{necromancer}, which should help them narrow the necromancer's location down.

\villages{Fallen Traders}

\begin{boxtext}

	A dead caravan lies ahead, with dead horses in front, and dead men at the side.  Every wagon, person and horse has been filled with arrows.

\end{boxtext}

Here the bandits have been again, and removed all the goods they could -- food, beer, clothing, and some swords.

The Night Guard still look for them, but they appear randomly, and then disappear into the depths of the forest.  This time they've been careless with their tracks.  The party can follow them with a Wits + Survival roll, TN 8, if they have the time and supplies for a two-day journey.

\villages{The Showdown}

The immortal bandits were in the area.  They have likely heard of the party by now and decided to follow.  Play the encounter below, and then have the bandits attack right after.

\humansoldier

Move this encounter to town, if the bandits survive.

\town{Bandits Caught}

\begin{boxtext}

	Hear, ye!  Hear all!

	Bandits who roamed the highways, lead by a man known as \gls{banditking}, have been apprehended.  The leader shall be drawn and quartered by week's end, and his companions hanged that night.

	Bakers are henceforth forbidden from purchasing the flour of the Quennome region, and any found doing so will be charged with consorting with elves.

	The temple invites any charismatic men or women to aid the festivities, as playwrites and actors are required for the upcoming festivities.

\end{boxtext}

\Gls{banditking} will not be killed by law enforcement.
Those in \gls{pig} will inform \gls{sewerking} long before, and the rescure will commence as the bandits in the sewer storm the guards' holding.
Meanwhile, if \gls{necromancer}'s lair survives, the other bandits await instructions there.

The only way for the characters to secure \gls{banditking}'s demise it to watch the guards' station all night.
If they do so, ten bandits stage an attack during the night.

\npc{\G}{Red Bandits}

\humansoldier

\sidequest{Desperate Measures}\label{desperatemeasures}

\gls{nurabaron} has come under a horrible nura curse, turning him into an ogre.
He has lost most of his mind, and must eat constantly.
His servants pity him, and continue to carry out his orders, which mostly include bringing him more and more food.  They know that if he's dispossessed, they'll all be out of work, and have to join the Night Guard, or worse.

As a result of the local starvation, many of the local peasants have turned to banditry.

\villages{Your Money or Please Don't Hurt Me}

Bertrand is healthy enough, but the rest of his crew of eight men look nearly emaciated.
They demand silver, or at least copper, but then quickly settle for any rations the characters might have.

\begin{boxtext}

	A single arrow hits the road ten feet in front of you with a dull thudd.
	A man stands up from the bushes nearby saying ``Stand forth, and deliver!''.

\end{boxtext}

If the characters refuse, the bandits might shoot, but they're easily intimidated.  If the characters attack, the bandits flee.

Of course, these aren't proper bandits.
This is the first robbery they've attempted.
If questioned, they explain that they're really farmers in the nearby town \gls{redfall}, and the local sheriff's been demanding steadily more and more meat and grain as taxes.
They don't have the strength to go on.

\NPC{\M}{Bertrand}{Pessimistic}{Squints}
\humansoldier

\npc{\G}{Emaciated ``Bandits''}
\person{1}{-1}{-1}{{0}{-1}{0}}{0}{1}{Beast Ken 2, Crafts 1}{\Dagger, (three have shortbows)}{}

If the characters investigate further, they may well end up at Redfall Keep.
In that case, have them stopped at the gate, and play out the encounter below with Nathaniel the Diplomat.\footnote{The None Starter.}
He won't let them in the keep, but he will promise all he can if the characters complete the mission he has for them.

This encounter then moves to town.

\villages{Wrong Direction Chickens}

\Gls{redfall} needs a lot of food to keep \gls{nurabaron}'s family fed, so they have started ordering more food.
Normally, villages feed the towns, but in this case the town is feeding the village.

\begin{boxtext}

	The road is speckled with light rain, and you pass by various traders en route.
	All of them are coming from town, so most trundle by with empty wagons, though one has a full cart of chickens in cages.
	The rain lets off just as the Sun sets, leaving everone damp.

\end{boxtext}

Slip in the fact that a trader is travelling with chickens away from town casually.
If the party notice, they can ask, and he'll tell them he's going to \gls{redfall} because he was paid a lot to do so.
Otherwise, just leave the clue dangling.

\town{The Nonstarter}

\begin{boxtext}

	A man wearing a fine, purple gown has been watching you from the side of the tavern for some time.

\end{boxtext}

Once approached, he explains (or if not approached, he approahces the PCs).

\begin{speechtext}

	You look like a capable bunch.
	I come with a mission from my lord, who shall remain nameless.
	A terrible curse has been cast on him, and he needs the services of \gls{forestpriest} to remove the curse.

	My patron will pay you a total sum of two hundred gold pieces in return for taking that priest, by any means necessary, to his castle.
	Ask for me in this tavern, and I will come to you as quickly as possible, and we will escort the high priest to my lord.

\end{speechtext}

\Gls{forestpriest} has gone on a personal pilgrimage some time ago.
Nobody knows where he is, and the characters have almost no chance of finding him.
The mission is a non-starter, because \gls{forestpriest} is not an easy character to find, but there's plenty of opportunity to get the characters in more trouble while they're looking.

Locating the priest can involve asking around town about where he planned to go, researching historical sites of interset, or asking local elves what happened to him.
No matter what the roll, the TN is 12.
A single roll prompts a forest encounter, and completing this as a resting action entails wading through three forest encounters.

Once the priest is located, he cannot actually `cast magic' and cure the \gls{nurabaron}.
The cure will involve starving the baron of all food until the magic disipates, and it is possible the baron will starve during this time.

If the characters convince Nathaniel that they can cure the baron, one must roll Intelligence + Medicine at TN 12.

\NPC{\M}{Nathaniel the Diplomat}{Practical}{Scratches nose}\label{nathaniel}

\humandiplomat

\town{The Town Crier}

Elliot the crier knows nothing more than he's said.  A number of townsfolk quickly decide to take up arms and slay the local monster, hoping the ransack his house and loot anything of value.

Of course, the only way to put a stop to this is for the characters to find the local priest and bring him to the keep before the angry mob arrive, or convince the mob that they have already cured \gls{nurabaron}.

The characters can stop trouble before it starts with a Charisma + Academics roll, TN 12.
Otherwise, they need to be at the keep to interfere with the growing mob.

\begin{boxtext}

	\begin{verse}

		Hear ye! Hear ye!

		Oi! I said ``Bloody well listen!''

		The current price of dwarvish coin is to be lowered by a tenth of the current value.

		Guards are no longer allowed to urinate in public.
		Guards caught urinating in public may be reported to the local guard station.

		Honest work is to be found digging fortifications in the Wetlaw town.

		It can wait till I'm bloody-well finished, Margaret.  Shut it!

		Listen good to this one!

		\gls{nurabaron} of \gls{redfall} Keep has turned evil, become a depraved monster, and is to be killed on site.  His last known whereabouts is his own keep.  Within this establishment, his own staff may be killed on the basis that they harbour a criminal.  All goods found therein are considered legal property by the finder.

	\end{verse}

\end{boxtext}

\vfill\null

%\subsection[The Message]{The Message}\label{message}
%
%\villages{The Dead Courier}
%
%\begin{boxtext}
%Conversation stops abruptly when you see death on the road ahead.  A caravan of four carts lies dead.  Ten bodies and three horses, all filled with arrows.  Perhaps one hundred arrows in total.  A little at the side of the road, a man lies bleeding from his arm, with a scroll still clutched in his hand.
%\end{boxtext}
%
%The messenger's name is Tobias of \gls{redfall}.  As the party arrives, he says ``Take this to \gls{redfall} keep'', then falls silent.
%
%Keeping him alive requires a Wits plus Medicine roll at TN 10, but even if he lives, he's in no state to do anything.
%
%The message has an unknown seal, belonging to no family in the area.  The characters know they should not look at it, but if they do, it reads:

\end{multicols}

\section{Locations \& People}

\includesvg{images/Dyson_Logos/redfall.svg}\label{redfall_map}

\subsection{\glsentrytext{redfall}}

\begin{multicols}{2}

The villagers in \gls{redfall} have become tired of armed tax collectors.  If the characters enter the area, they will find peasants laying an ambush for them.

\begin{boxtext}

	As you step onto the bridge, you see a dozen men walking behind you with large bags covering their heads.  In front, another ten gather as well.  Each carry a large sack, then one reaches in and plucks out a rock.

\end{boxtext}

The farmers don't want to kill the characters -- just scare them off; but it will take some quick talking to stop the stoning before it begins.  The party may roll Wits + Empathy, TN 11.

The villagers have no idea that \gls{nurabaron} has been tainted with dark magic, they only know they haven't seen him in a long time, and that he has increased taxes, and takes almost all his taxes in food, back to the keep.

\npc{\G}{20 Angry Farmers}

\humanfarmer

Once the fight is over, the leader of the group -- Matt -- will come forward, and may even apologize for the misunderstanding if the party have a good explanation as to why they're in \gls{redfall}.

\end{multicols}

\subsection*{\glsentrytext{redfall} Keep}

\includesvg{images/Dyson_Logos/redfall_keep.svg}\label{redfall_keep_map}

\begin{multicols}{2}

\setcounter{list}{0}

The keep contains a conspiracy: everyone in there knows that they harbour monsters, but they also know that those monsters can be cured, and that if they tell anyone, they'll lose their employment, and will either have to beg or join the Night Guard.

The guards and servants have rationalized this as best they can, and continue to push for more and more food to be brought to the keep, while defending their lord zealously.

\mapentry{Guard Tower}

This tower has two guards with bows, ready to rain arrows down on any intruders.

\npc{\G}{2 Archers}

\humanarcher

\mapentry{Courtyard}

This open courtyard makes archery an easy job.  Anyone outside can easily be picked off if they do not have adequate shielding.

\mapentry{Toilet}

This spacious lavatory comes with a runoff to the river outside.
It also hosts arrow slits to fire at intruders on the river.
Small characters (with Strength of 0 or less) can fit through the gap from outside, if they are not wearing armour.

\mapentry{Stables}

One of the horses here was accidentally hit by some of the ogre-dust which affected \gls{nurabaron} and his family.  It ate the others, but despite the men protesting, \gls{nurabaron} has insisted on keeping it alive and very well fed.

\npc{\N\C}{Janus, the Demonic Horse}

\nurahorse

\mapentry{Food Storage}

The dwindling food supplies are kept here.  They build up to monstrous amounts, then get hurriedly eaten within a day or two.  Barrels of meats, stacks of bread, and multiple bottles of wine -- nothing lasts long.

\mapentry{Servants' Quarters}

This place holds various cots, chests, and washing equipment.  At present only six members of staff remain -- two cooks, two butlers, one displomat,\footnote{See page \pageref{nathaniel} for Nathaniel the diplomat.} and James the tax collector.

\mapentry{Benjie's Room}

\Gls{nurabaron}'s youngest son Benjie is only a toddler.
He doesn't understand his condition, and has no self control.
This makes him dangerous, so the family have decided to lock him in this little room.
Characters on the river or anywhere near this room can hear intermitent banging, and infant-like gurggling.

\NPC{\M\N}{Benjie -- ``the Toddler''}{Playful}{Fingers in mouth}

\deephobgoblin

\mapentry{The Great Hall}

\begin{boxtext}

	You open the door to a massive table full of opulant food, and more food stacked on top.  Layers of bloody bones poke out the side of three layers of filthy dishes, lining the bottom of the feast.  At the head of the table, sits \gls{nurabaron}, in a meat-stained dressing gown.

\end{boxtext}

A heavy wooden door guards the entrance well, and a gate sits in front of that.  Lifting the gate takes a Strength + Athletics roll, TN 10, but the locks on the door don't give much resistance; players can pick them with Intelligence + Larceny, TN 6.  The bars are slim enough that anyone with a Weight Rating of 5 or less can just about squeeze through.

The hallways contains a long, messy dining table, covered in scraps of bones from all manner of animal.  The side alcoves contain little beds for guards to sleep in.  Four guards are in the area at any one time.

The baron can usually be found dining here, shouting at guards about some imagined insult, or just terrifying people for the fun of it.  Nathaniel, his trusted advisor, is the only one who can calm him down when he gets into a rage.

\NPC{\M\N}{\glsentrytext{nurabaron}}{Hungry}{Wide eyed}

\person{5}{0}{4}{{-4}{-2}{-4}}{0}{1}{Academics 1, Beast Ken 1, Deceit 2}{\greatsword}{}

\mapentry{The Old Storage Shelves}

Here the servants rest.
The armoury was once staffed by men at arms, but now has nothing but ogre-powder and other magical items from the \textit{nura depths}.
Anyone disturbing the contents of this room must make a Dexterity + Crafts check, TN 7, or risk toppling something and spilling the powder everywhere, which will affect the characters as per the Saurecanta sphere (page \pageref{saurecanta}); specifically, they get +2 to Strength and Speed, but -4 to Intelligence and Charisma.
This powder is not nearly so potent as the charm cast upon \gls{nurabaron} and his family, so they will recover at the end of the scene.

\mapentry{The Hidden Family}

In the upper floor, \gls{nurabaron}'s wife Marjorie stays with her four children -- two boys and two girls.  All are a little too dangerous to be trusted wandering alone, so only one is let out of their room at a time.

\npc{\G\N}{Marjorie \& Her Daughter}

These two attack fiercely, but retreat if any bargains have been made after a single round.

\ogre

\npc{\G\N}{Richard \& Jim}

The two brothers were never good at listening, and will attack with abandon, unless Marjorie is nearby to talk them down.

\ogre

\end{multicols}

