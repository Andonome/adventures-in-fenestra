% !TeX root = ../main.tex
\setcounter{encnum}{1}\renewcommand{\encsymbol}{\ding{170}}

\section{Summaries}

\subsection[Here and No Farther]{Here and No Farther -- \encnum}

\gls{baron} wants to expand his domain into a nearby forest, but \gls{forestpriest} has other plans.
Page \pageref{herenofarther}.

\begin{enumerate}
	\item{The characters find an unusual deer.}
	\item{(Forest) The characters stumble upon a garrison of masons with a large supply of malachite.}
	\item{(Town) They find an unusual sheep.}
	\item{(Town) \glsentrytext{forestpriest} is labelled a criminal.}
\end{enumerate}

\subsection[The Necromancer's Pets]{The Necromancer's Pets -- \encnum}
A necromancer slowly builds an army of the dead.
Page \pageref{necromancerspet}.

\begin{enumerate}
	\item{A crow raises a body from the dead.}
	\item{(Forest) The characters are attacked by an undead horde.}
	\item{A band of traders have been killed, but followers of Qualme were left alive.}
	\item{Night of the dead.}
	\item{\ding{106} The necromancer assaults a village.}
\end{enumerate}

\subsection[Immortal Bandits]{Immortal Bandits -- \encnum}
A bandit ring uses necromantic magic to change location quickly on undead horses.
Page \pageref{immortalbandits}.

\begin{enumerate}

	\item{A local village has had all its feast supplies stolen by bandits, and the grave of a dead hero raided.}
	\item{Bandits hide at the side of the road, but don't trouble the characters.}
	\item{A local tracker returns as a zombie, wearing a magical necklace.}
	\item{A wagon of traders are found stuffed with arrows.}
\end{enumerate}

\subsection[Desperate Measures]{Desperate Measures -- \encnum}
A local Villagemaster and his family have been transformed into a nura monster, and they are eating enough to starve out the local village.
Page \pageref{desperatemeasures}.

\begin{enumerate}

	\item{\ding{106} Pathetic bandits assault the characters.  They were villagers but have turned to banditry out of hunger.}
	\item{A wagon with chickens is journeying \emph{from} town.}
	\item{\ding{106} (Town) A diplomat requests the characters track down a priest to cure a local lord of a curse.}
	\item{\ding{106} (Town) A town crier says the priest is back, and that a bounty has been called on a local lord.}
\end{enumerate}

%\subsection[The Message]{The Message -- \encnum}
%\gls{nurabaron} has made a deal with a mysterious figure from deep underground.  He wants to raise and army and disposess the crown.  He has many friends already, but knows he needs to gain assassins and hidden powers if he wants to play this right, so he's bargained for magical items.

%The deal so far is clear -- a limited shipment of food is to be delivered.  Some of it has been delivered already, and the rest is calculated in a book \gls{nurabaron} keeps hidden in his room.
%
%Page \pageref{message}.
%
%
%\begin{enumerate}
%
%\item{The bandits robbed the courier blind, but his message remains in his hand.} % The message is about receiving a magical item.
%\item{Someone murdered an entire household, and monsters sit in the atic.}
%\item{\ding{106}\ding{106}  Villagers have turned into nura monstrocities.}
%\item{A nobleman insults and taxes the characters.}
%\item{\ding{106}\ding{106}  Bandits eat a trader's corpse -- all have turned into nura.}
%
%\end{enumerate}

\setcounter{encnum}{1}
\section{Encounters}\index{Encounters!Villages}
Centuries ago, men created a logging city, and the elves destroyed it.  Recently, \gls{baron} has decided to reclaim it, though nobody knows exactly where it is.  He sent a dozen trackers to find the place, but \gls{forestpriest}, thought it a bad idea, and has been turning anyone who finds or hunts for the place, into animals.

\subsection[Here and No Farther]{Here and No Farther -- \encnum}\label{herenofarther}

\begin{boxtext}

	A startled deer stops in front of you, then just stares.

\end{boxtext}

Yesterday, Jade the tracker came to \gls{lostcity}, but \gls{forestpriest} didn't like the intrusion, so he turned the intruder into a deer.  Jade fled instantly.

If the characters begin to question the odd behaviour of the deer, they can make a Wits + Beast Ken roll to notice that something is wrong, TN 10.

Anyone attempting to dispel the magic must roll at TN 14, so in all likelihood the deer will remain a deer.

\character{Jade the Tracker}
\person{1}{1}{0}{{0}{0}{1}}{0}{1}{Beast Ken 2, Empathy 1, Survival 2}{Nothing, not even clothing.}{}

If returned to human form, Jade can instantly identify \gls{forestpriest} as the man who turned him, though in deer form he will have a lot of trouble expressing himself, and can do little more than show distress around the miracle worker.

The characters may decide to visit \gls{forestpriest} in order to have the spell lifted, but \gls{forestpriest} simply tells them that this is an ordinary deer, enchanted to think that it is a man.

Move this encounter to the Forest.

\subsubsection{The Green Tower}

\begin{boxtext}

	In the deep forest, where there should really be nothing, you see a little tower, and a man hanging out the window, painting it green.  In another day, you're sure you wouldn't have been able to see it at all.

\end{boxtext}

Deep in the forest, \gls{baron} has commissioned masons to start building a single tower from the fallen stone of \gls{lostcity}. This is the first forway into the deep wilderness.  The top is to be disguised by painting is green (the color is made by malachite, which can be expensive).

The players may watch the building work and report back, or confront the builders, or any number of things.  It's unlikely, at this stage, that they find the secret portal to catacombs of the old Temple of Qualm\"{e}.

\gls{nightguard} leads the group.  He knows he's meant to keep the place secret, but the masons have no plans to harm the characters unless they are attacked.

Move this encounter to the town.

\subsubsection{Rogue Sheep}

Run this encounter in Town, at the same time as the encounter below it.

\begin{boxtext}
	A man in ragged clothing, missing a couple of fingers and a couple of teeth, chases a sheep down the road.  A woman shouts out \emph{``Oi, Trevor! That ain't your sheep!''}.

	\emph{``Well whose bloody is it then, bitch?!''}, the ragged man snarls back.

	The sheep runs into a tavern, and Trevor runs in after it.
\end{boxtext}

Yesterday, \gls{baron} and a number of his men met to discuss further building inside the deep forest, and hunting for \gls{lostcity}.  One of the maids who worships at the temple of Laique on the Town's outskirts heard this and told \gls{forestpriest}.  \Gls{forestpriest} tracked down the main architect, Darren, cornered him in an empty ally, and turned him into a sheep with his polymorphing ability.

The sheep goes into the \gls{pig} because it's the only place of safety he knows.  The sheep is obviously sentient, if anyone bothers to ask him clear questions.

Before the characters can decide what they're doing, the next encounter here plays.

\subsubsection{\glsentrytext{forestpriest} Exiled}

\Gls{forestpriest} has been found to be a criminal after \gls{captain} stalked him, and found him turning a mason into a pig.  The pig ran away, and was murdered and eaten before \gls{captain} could find him.

\begin{speechtext}
	Hear, ye! Hear, ye!

	A criminal from the Whiteland former nobility has been found, and shall be executed tomorrow, noon, together with three thieves and one man who blasphemed the crown.

	\Gls{forestpriest}, former high priest of Laique, has been caught using black arts, including cannibalism, and now carries a total reward of 200 silver pieces living, and 300 silver pieces dead.

	All elves in town must carry with them a letter of registration, stating their business in town, and their current lodgings.  Reporting an unregistered elf brings a 50cp reward.

\end{speechtext}

Elvish characters may have a difficult time with the registrations, since everyone in town will ask them for documentation in the hopes of turning them in for a reward.

The criminal of the Whiteland Nobility is someone in hiding, who's come to meet with \gls{banditking} but would rather hang than give up his kin's whereabouts.

\subsection[The Necromancer's Pet]{The Necromancer's Pet}\label{necromancerspet}

\Gls{necromancer} has grown increasingly paranoid, so he wants to gather an army of the dead.  He starts by enchanting a bauble which raises people as zombies, then enchants crows, so they can bring and use the medalions on any corpses they find.  Once he has gathered a large enough army, he begins assaulting villages.

\subsubsection{The Crow}

Play this encounter at the same time as the one below.  If a corpse appears during the encounter, that's where the crow goes.  If no corpses appear, the characters find a trader by the road, robbed by unknown bandits, and the crow on top of his body.

\begin{boxtext}
	A crow lands upon the corpse and stares at the vacant eyeballs, before landing a sturdy peck on the left one.

	A little glimmer around its neck shines.  The crow wears a medalion, and a moment later it takes off, leaving the right eyeball undigested.

\end{boxtext}

A moment after the crow has flown away, the corpse animates, and attacks the characters.  Even if this one doesn't join the characters, \gls{necromancer} has gathered an army of 50 zombies by looting the dead.

\monster{Zombie}
\zombie

This encounter moves to the forest area.

\subsubsection{The Undead Hord}

\begin{boxtext}
	Crackling sticks indicate someone walks closeby, and a moment later indicates many people.  A full procession approaches, but no voices can be heard.
\end{boxtext}

A full horde of twenty zombies approaches from the distances.  The party makes a Wits plus Vigilance roll, TN 6, to notice the undead while they are still 10 squares away.  Each roll on the margin indicates an additional 10 squares to notice the horde, so rolling 10 means 50 squares' distance.

The horde has wandered away from its necromancer, while the necromancer was distracted.  There is no way to use them to find where the necromancer lives, as they have no desire to return to it.

This encounter then returns to the Villages.  While these ones have run away, the necromancer has gathered a full army of 100 zombies.

\subsubsection{The Survivors}

\begin{boxtext}
	On the horizon, four humanoid sillhouettes stumble forward silently.  Once they see you, they start running towards you.
\end{boxtext}

Last night an undead horde pulled apart an entire village.  The undead were directed well by the necromancer, and nobody was to escape.  However, four people were evident followers of Qualme, god of death.  Two wore medalions to commemorate departed loved ones, and another two gave prayer to the God of Death when the undead came.  As a result, they were spared.

Rupert, Jake, Sarah and Eliza cannot explain why they were spared -- they only know the udnead never touched them while they ate the read of the village.

If the characters track the zombies, they find it rains en route, making tracking more difficult.  However, a Wits plus Survival roll, TN 10, will get them there.

By this time the necromancer has gathered an army of 200 undead.

\subsubsection{Night of the Dead}

The necromancer has gathered an army of a full 400 undead, and has decided to take most of them to a nearby village.  Most are not under his control -- he uses his Necromancy sphere to control some, and the rest usually follow.

\begin{boxtext}
	The village falls quiet at night, except for shuffling feet as people try to be quiet going to the toilet at night, or chattering about local town gossip.

	A man in the distance tells his child off for going out at night into the forest with his friends.
\end{boxtext}

The necromancer has decided it's time to gain more power.  He has the village surrounded.

His tactics are to create a full ring of undead, and pull it tighter.  He stays around the periphery, and pulls them in slowly.

For tactical purpose, divide the village into four quadrants -- perhaps `the well', `the hallway', `the fields', and `the shrine', or whatever fits the village the characters have ended in.

Each quadrant is attacked by 50 zombies, while the necromancer stays around the outside, picking off anyone who tries to escape.  He starts with hit hunting bow -- his undead sight allows him to spot people in the dark with ease.  After that, he uses magic, starting with curses, then invocation magic.

\character{\Glsentrytext{necromancer}}
\person{0}{0}{0}{{2}{0}{-2}}{2}{1}{Academics 2, Deceit 1, Empathy 1, Survival 1, Tactics 2\Path{Devotion}{Fate 2, Invocation 2, Metamagic 4, Necromancy 3}}{\longsword}{\mana{6}}

\subsubsection{200 Zombies}

\zombie

The villagers will fight with the characters.  If the characters simply enter battle, without further complications, you can use this as a guide for the assault, where each character has move undead attack each round:

\begin{rollchart}
	Round 1 & 5 zombies attack. \\

	Round 2 & 3 zombies attack. \\

	Round 3 & 2 zombies attack. \\

	Round 4 & Nearby houses go down, torn apart by the dead. \\

	Round 5 & All houses a broken, and the dead invade. \\

	Round 6 & Those already dead rise again as the necromancer completes his spell. \\

	Round 7 & The dead retreat at the necromancer's order, and any dead villagers return with him. \\

\end{rollchart}

Once the battle's over, the village will most likely be mostly destroyed, and \gls{necromancer}'s army much larger.  The villagers will shake, huddle together, and most consider moving.

The encounter repeats until the party slay \gls{necromancer}.  The characters can complain to local nobles, but they note that they are not allowed a standing army, so they can do nothing about the situation.  The party can write to the king, but they will be lucky to receive anything but stock letters from beaurocrats stating that the situation has complications.

\subsection[Immortal Bandits]{Immortal Bandits}\label{immortalbandits}

\Gls{banditking} and his men are a bandit ring with a difference.  Most bandits have to raid local areas, until trackers in the Night Guard eventually catch and kill them.  \gls{banditking} and his team have rings which give them partial powers of undeath, and ride undead horses.  The horses walk slowly, but never tire, so \gls{banditking}'s band can raid villages for many miles around before retiring to an impregnable keep.\footnote{See page \pageref{necromancers_lair} for more on their ugly living conditions.}

\subsubsection{The Feast is Cancelled}

\begin{boxtext}

	The village looks like some kind of inverted funeral.  A woman stands in a beautiful beige dress at a wooden alter to V\'{e}re, consoled by a man wearing the traditional bright wedding hat.  Three dead men lie at the altar's feet, including the a priest.  The village has feasting tables out, but all sit empty.

	A crow attempts to land on one of the dead men, but a local man lunges at it, then gives chase, as if trying to follow the crow into the sky.

\end{boxtext}

Earlier, \gls{banditking} and his men came, took all the food they could, murdered three men who drew swords, then left.

The bandits are ten miles away, and can still be tracked with a Wits + Survival roll at TN 10.

Characters with a good eye (Wits + Crafts, TN 11), can spot that the crow has a medalion around its neck.

\subsubsection{Hidden Eyes}
\begin{boxtext}
The road through over to the hamlet is quiet, with only rustling trees, and a crow cawing in the distance.
\end{boxtext}

\gls{banditking} and twenty men hide in the dense trees on a hill, a mile away from the road.  They watch, quietly, for traders.  A single man, Engel, sits closer to the road, waiting to call like a pidgeon if someone rich wanders onto the road, or call like a crow if an armoured troop wander along.

Nobody will bother the characters, unless they look both rich and unarmed.

Spotting the fake call requires a Wits + Beast Ken roll, TN 10.

If the characters capture Engel, he claims to be a hunter.  If they come for the rest o f the bandits, they flee on horseback.

\character{Engel}

\humansoldier

\subsubsection{The Dead Tracker}
Anderson was hired to track down the bandits.  However, the Necromancer's pet crow spotted him, and informed the bandit leader, \gls{banditking}.  He was caught, and killed, then left; but one of the necromancer's crows turned him into a zombie.  Instead of going to the necromancer, he wandered aimlessly.

The party may notice by his green clothing that he serves in the Night Guard.

\zombie

\subsubsection{Fallen Traders}

\begin{boxtext}
	A dead caravan lies ahead, with dead horses in front, and dead men at the side.  Every wagon, person and horse has been filled with arrows.
\end{boxtext}

Here the bandits have been again, and removed all the goods they could -- ffood, beer, clothing, and some swords.

The Night Guard still look for them, but they appear randomly, and then disappear into the depths of the forest.  This time they've been careless with their tracks.  The party can follow them with a Wits + Survival roll, TN 8, if they have the time and supplies for a two-day journey.

\subsubsection{The Showdown}
The immortal bandits were in the area.  They have likely heard of the party by now and decided to follow.  Play the encounter below, and then have the bandits attack right after.

\humansoldier

Move this encounter to town, if the bandits survive.

\subsubsection{Bandits Caught}


\begin{boxtext}
	Hear, ye!  Hear all!

	Bandits who roamed the highways, lead by a man known as \gls{banditking}, have been apprehended.  The leader shall be drawn and quartered by week's end, and his companions hanged that night.

	Bakers are henceforth forbidden from purchasing the flour of the Quennome region, and any found doing so will be charged with consorting with elves.

	The temple invites any charismatic men or women to aid the festivities, as playwrites and actors are required for the upcoming festivities.

\end{boxtext}

\Gls{banditking} will not be killed by law enforcement.  Those in \gls{pig} will inform \gls{sewerking} long before, and the rescure will commence.  Meanwhile, if \gls{necromancer}'s lair survives, the other bandits await instructions there.

\subsection[Desperate Measures]{Desperate Measures}\label{desperatemeasures}

\gls{nurabaron} has come under a horrible nura curse, turning him into an ogre.  He has lost most of his mind, and must eat constantly.  His servants pity him, and continue to carry out his orders, which mostly include bringing him more and more food.  They know that if he's disposesed, they'll all be out of work, and have to join the Night Guard, or worse.

\subsubsection{Your Money or Please Don't Hurt Me}

\begin{boxtext}
A single arrow hits the road ten feet in front of you with a dull thudd.  A man stands up from the bushes nearby saying `Stand forth, and deliver!'.
\end{boxtext}

Bertrand looks healthy enough, but the rest of his crew of eight men look nearly emaciated.  They demand silver, or at least copper, but then quickly settle for any rations the characters might have.

If the characters refuse, the bandits might shoot, but they're easily intimidated.  If the characters attack, the bandits flee.

Of course, these aren't proper bandits.  This is the first robbery they've attempted.  If questioned, they explain that they're really farmers in the nearby town \gls{redfall}, and the local sherrif's been demanding steadily more and more meat and grain as taxes.  They don't have the strength to go on.

\character{Bertrand}
\humansoldier

\subsubsection{Emaciated ``Bandits''}
\person{1}{-1}{-1}{{0}{-1}{0}}{0}{1}{Beast Ken 2, Crafts 1}{\Dagger (three have shortbows)}{}

This encounter then moves to town.

\subsubsection{The Nonstarter}

\begin{boxtext}
	A man wearing a fine, purple gown has been watching you from the side of the tavern for some time.
\end{boxtext}

Once approached, he explains (or if not approached, he approahces the PCs).

\begin{speechtext}
	You look like a capable bunch.  I come with a mission from my lord, who shall remain nameless.  A terrible curse has been cast on him, and he needs the services of \gls{forestpriest} to remove the curse.

	He will pay you a total sum of two hundred gold pieces in return for taking that priest, by any means necessary, to his castle.  Ask for me in this tavern, and I will come to you as quickly as possible, and we will escort the high priest to my lord.
\end{speechtext}

\gls{forestpriest} has gone on a personal pilgrimage some time ago.  Nobody knows where he is, and the characters have almost no chance of finding him.  The mission is a non-starter, but there's plenty of opportunity to get the characters in more trouble while they're looking.

\character{Nathaniel the Diplomat}
\humandiplomat

\subsubsection{The Town Crier}
\begin{boxtext}
	\begin{verse}
		Hear ye! Hear ye!

		Oi! I said ``Bloody well listen!''

		The current price of dwarvish coin is to be lowered by a tenth of the current value.

		Guards urinating in public may be reported to the local guard station.

		Honest work is to be found digging fortifications in the Wetlaw town.

		It can wait till I'm bloody-well finished, Margaret.  Shut it!

		Listen good to this one!

		\gls{nurabaron} of \gls{redfall} Keep has turned evil, become a depraved monster, and is to be killed on site.  His last known whereabouts is his own keep.  Within this establishment, his own staff may be killed on the basis that they harbour a criminal.  All goods found therein are considered legal property by the finder.


	\end{verse}
\end{boxtext}

Elliot the crier knows nothing more than he's said.  A number of townsfolk quickly decide to take up arms and slay the local monster, hoping the ransack his house and loot anything of value.

Of course, the only way to put a stop to this is for the characters to find the local priest and bring him to the keep before the angry mob arrive.

\subsection[The Message]{The Message}\label{message}

\subsubsection{The Dead Courier}

\begin{boxtext}
Conversation stops abruptly when you see death on the road ahead.  A caravan of four carts lies dead.  Ten bodies and three horses, all filled with arrows.  Perhaps one hundred arrows in total.  A little at the side of the road, a man lies bleeding from his arm, with a scroll still clutched in his hand.
\end{boxtext}

The messenger's name is Tobias of \gls{redfall}.  As the party arrives, he says ``Take this to \gls{redfall} keep'', then falls silent.

Keeping him alive requires a Wits plus Medicine roll at TN 10, but even if he lives, he's in no state to do anything.

The message has an unknown seal, belonging to no family in the area.  The characters know they should not look at it, but if they do, it reads:

\section{Locations \& People}

\subsection{The Bandits}

\gls{banditking} was the son of a nobleman in Whiteland, but his family have been killed, so he became an outlaw.  Last year, he and his men ambushed some guards working for Redearth and took magical rings from their corpses.  His men called him \gls{banditking} ever since.

\character{\glsentrytext{banditking}}
\person{2}{2}{0}{{1}{-1}{2}}{0}{3}{Academics 1, Empathy 1, Deceit 3, Survival 3, Tactics 2}{\longsword 5gp wrapped in cotton whool, knife}{}

\subsection{\glsentrytext{redfall}}
\vspace{-1cm}

\includesvg{images/Dyson_Logos/redfall.svg}\label{redfall_map}


The villagers in \gls{redfall} have become tired of armed tax collectors.  If the characters enter the area, they will find peasants laying an ambush for them.

\begin{figure}
	\subsection{\glsentrytext{redfall} Keep}
\vspace{-2cm}

\includesvg{images/Dyson_Logos/redfall_keep.svg}\label{redfall_keep_map}

\end{figure}

\subsubsection{The Lower Floor}

A heavy wooden door guards the entrance well, although the servants' quarters door round the back has rotted, and the servants `lock' is only by tying a rope.

The entire hallway past the entrance look barren, as most of the goods have been sold by \gls{nurabaron} already.

The place contains the usual rooms -- a parlour for entertaining guests, a kitchen, and a few rooms.

Within these various rooms, the baron, his wife, and five children rest.  All have been turned into ogres.

\character{Marjorie and Five Children}

\ogre

\character{Benjie -- ``the Toddler''}

\deephobgoblin


\subsubsection{The Upper Floor}

Here the servants rest.  The armoury was once staffed by men at arms, but now has nothing but Ogre Powder and other magical items from the \textit{nura depths}.  Anyone disturbing the contents of this room must make a Dexterity + Crafts check, TN 7, or risk toppling something and spilling the powder everywhere, which will affect the characters as per the second level of the Saurecanta sphere (page \pageref{saurecanta}); specifically, they get +2 to Strength and Speed, but -1 to Intelligence and Charisma.  This powder is not nearly so potent as the charm cast upon \gls{nurabaron} and his family, so they will recover at the end of the scene.

\character{\glsentrytext{nurabaron}}

\person{5}{0}{4}{{-4}{-2}{-4}}{0}{1}{Academics 1, Beast Ken 1, Deceit 2}{\greatsword}{}

% Add a village encounter -- as soon as the characters have gained ogre strength, the local villagers decide to raid the place.

