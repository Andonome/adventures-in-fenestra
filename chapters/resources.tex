% !TeX root = ../main.tex

This section is not a `how to GM guide'.  You can find excellent generic GMing guides in abundance on-line.  This is a collection of tips for GMs to work better with the First Blood rules.

\section{Basic Prep \& Play}

The basic tools of the Games Master must begin with with obvious -- $4D6$ per player with multiple $D6$ colours so players can differentiate their Damage dice from their Action dice.  Next, of course, one character sheet and pencil for each of the players.  Since this can be a lethal game, especially for new players, consider adding a few `just in case' character sheets.

Now for the special stuff: make sure you have a few coins for your players.  By far the easiest way for players and GM alike to track Initiative is to stick a coin on their current Initiative Score's number and move it down when the action is done to the next Initiative point where that character is acting.  As a GM, it's always good to have at least 3 different types of coins.  Let's say you're orchestrating a battle with a hobgoblin leader, some hobgoblin troops and a nura spider. Assign each one a coin and make a little mnemonic -- the spider has dark skin so it gets the little copper penny.  The hobgoblins get the silver coin to represent their use of weapons, and the largest coin goes to the hobgoblin leader.  Don't worry about the players' Initiative -- they'll keep track of their own characters as you shout out where on the Initiative tree you are.

Coins should also be used when assigning the Combat Skill.  The character sheets contain a large space in the middle where players can add bonuses to their Combat Factors rather than attempting to remember where everything was placed.

Coins can even be used to keep track of FP and Fatigue Points as they change so often.  Simply keep a pot with two different colours (or sizes) of coins and then place them on the character sheet at the appropriate location.

\subsection{Experience Points \& the Discount}

Standing alone against a towering ogre is a nightmare, but three warriors standing against three ogres can be much easier.  A battle against thirty goblins can really take its toll, but three different battles against ten goblins can be child's play.  To represent this, we have \textit{the XP Discount} -- a price you pay for every member of the party.  This was covered a little in the core rulebook, but let's have a better look here, since the GM will be doing the legwork.

For every member of the party, that many points are deducted from one monster's XP value (to a minimum of 0).  If the party has two members, the first two monsters have 2 XP deducted from their total value.  If the party has five members, the first five monsters have 5 XP deducted from their total.

If a single warrior fights a dragon worth 22 XP, then the warrior receives 21 XP, because 1 XP is always removed from the total.  If he fights 10 ghouls worth 2 XP each, then he receives ($10 \times (2 - 1) =$) 10 XP.

However, if five characters are fighting the 10 ghouls together, they each deduct 5 XP from a single monster.  The first five ghouls are worth nothing, because each net ($2 - 5 = $) 0 XP.  Only the last 5 ghouls count, bringing 10 XP in total.  Dividing this among 5 players, each receives 2 XP at the end.

If players need to discount multiple adversaries, they are counted from highest to lowest XP.

\subsection{New Players}\index{New Players}

For a quick start game with players unfamiliar with roleplaying systems or who simply want to deal with fewer challenging concepts, consider a few simplifications. This is not so much a rule change as withholding information so as not to overwhelm people.

At character creation, just have players roll some dice to make their characters.  It's far easier to explain `You are a gnoll' than `There are five major races.  Read their descriptions here and then select one'.  If anyone has a race in mind or wants to play a specific `class', then just allow that player to make their character `by hand' rather than by the dice, but the default option for new players should always be a randomly generated character -- it simplifies and speeds things up  immensely.

When it comes to the rules, don't mention Attack Stances -- just say that Dexterity increases a character's Evasion score without referencing a character's ability to fight with an Aggressive Stance.  Secondly, don't mention dividing the Combat Skill Bonus between any arrangement of Strike, Evasion and Initiative -- just let the players each add it to Strike. If a player is industrious enough to actually read the book and come across the rule, it should of course be allowed. But as long as the players are not optimizing their attacks by using the complete rules, remember not to do so yourself.

\subsection{Central Initiative}

The help people understand the tactical elements of the game, consider setting a central initiative track on the table, and have everyone place a token, model, coin, or whatever on their own Initiative number.

\section{Classes}

Players accustomed to class-based systems may request to play a particular class. Don't argue, just select one of the `class packages' from below.  An alchemist is just someone with spells, and a rogue is just someone with skills.  Once the game starts, players can spend their XP as they wish.

\subsection{Alchemist}\index{Alchemist}

Alchemists start with Academics 1, Invocation 2, Illusion 1 and MP 2.  If their Intelligence or Wits is below 0 then raise it by one level.  If not, buy a single 1st level Skill.

Their equipment is a notebook and writing equipment, camping equipment and a quarterstaff. They worship C\'{a}le.

\subsection{Bard}\index{Bard}

Bards begin with Performance 2, Academics 1, Empathy 1, Deceit 1, Larceny 1, Vigilance 1 and the first level of the Fate sphere.

Their starting equipment includes partial leather armour, a dagger, an instrument, a longsword, lantern, camping equipment, notebook and writing equipment, and 50' of silk rope.  They worship Alasse.

After time spent adventuring, many bards learn Song Magic in order to aid their party.

\subsection{Priest of V\'{e}r\"{e}}\index{Priest of V\'{e}re}

Priests of the god of honour begin with Fate 2, Academics 1, Medicine 1 and MP 4.

Their equipment is a quarterstaff, medical equipment, partial chainmail shirt and camping equipment.

After gaining XP, some adventuring clerics focus upon martial abilities, while others focus on prayer in order to work miracles.

\subsection{Rogue}\index{Rogue}

Rogues begin with Combat 1, 10 FP, Stealth 2, Larceny 1 and the Knack: Perfect Sneak Attack.  If they have a Body Attribute at -1, raise it by one level.  If not, purchase Crafts with a specialisation in Locks.

Their starting equipment is a dagger, Complete leather armour, a longsword, 50' of rope and lockpicking tools.  If they have the Craft Skill, they begin play with a throwing dagger.  They follow the Code of Acquisition.

\subsection{Warrior}\index{Warrior}

Warriors begin play with Combat 2, FP 10 and the Knack: Adrenaline Surge.  If the character has a single Body Attribute below 0 then buy it up a level; otherwise purchase the Skill Survival at 1st level.

Their starting equipment is partial chainmail, a longsword and a bucklar shield.  If they start play with the Survival Skill they also get camping equipment.  They follow the goddess Ohta.

\subsubsection{Paladins}\index{Paladin}

After progressing, particularly pious fighters often gain a level or two in Fate, allowing them to ask for Divine Guidance, curse enemies, or even gain blessings before going into battle.

\subsubsection{Rangers}\index{Ranger}

Fighters with an affinity for nature, such as elves, may pick up nature-related abilities, such as talking with animals, or even summoning mists.  Whether this comes through prayer, or inborn abilities which develop over time, a little magic on the side of a character can make for a formidable fighter.


\section{GM Suggestions}
I lied a little when I said this was not a `how to GM guide'.  Allow me a couple of suggestions that have worked well with First Blood.

\subsection{Fast Initiative \& Good Pacing}

Rather than writing each player's Initiative down and then running through people in order, it can be a lot simpler to ask each player to keep track of their own Initiative, with pencil and paper or just marking it with a die. Thereafter the Games Master simply needs to count down. If the players rolled an 8 for Initiative then it's quite possible one of them could be acting at 12, so that's where you start. You already know your monsters are at 9, so you begin.

\begin{quote}
`Twelve! The gnolls ready their weapons'

`Eleven, ten! They move forward, bearing their yellowed teeth'

`Nine! Snarls abound as they speed up to a rush'
\end{quote}

Nothing has actually happened by this point, but it sets the scene nicely.  Players can find a handy initiative track at the side of their character sheet, counting from 18 downwards.

\begin{quote}
``Nine'', one of the players shout.  I'm going at nine.  I move to protect Max.

``Two gnolls go for you, another two go for Amelia.  Roll to defend at TN 11''

\end{quote}

The initiative continues down quickly at all times, and the count always provides a sense of urgency.

\subsection{Rollplay Before Roleplay}

It's hard to play `the social character'.  You put all your XP into a high Charisma score because you want to build alliances and understand people, then the GM asks you to roleplay the encounter and all that comes out is your natural stutter.

It's also hard playing a non-social character.  You have been lumped with a character with a Charisma Penalty of -4 and by all the gods you intend to roleplay it, so it's time to ask the town master which lady he stole his robe from and then wipe your mouth with the tablecloth.  But the other players are not impressed; all they can see is someone intentionally ruining the encounter rather than the fun-loving, amazing improviser that you are.

Consider the following solution: tell the players that if they wish to speak, they must roll Charisma plus Empathy or Wits plus Whatever, then set the TN for the encounter.  Getting information from the drunken patron of a temple of Alasse might be TN 4 while getting a noble to stop and give everyone a hand might be TN 10.  The player should not declare the result but make a mental note of the roll's Margin.  If the Margin is high, they should confidently roleplay someone saying just what the situation appears to demand.  On the other hand, if the roll was not only a failure but had a high Failure Margin, they should attempt to roleplay the worst kinds of insults -- perhaps because the character is genuinely mean-spirited, perhaps because they are making persistent, accidental faux-pas.

Picture the scene: the PCs have successfully entered the Shattered Castle.  A noble paid them in and said that they were to be taken to the Bearded Mountains, but in fact they want to go to Quennome on secret business.  The players need to convince the guard -- their escort -- to take them to the other portal.  The TN is 12.

If they roll a Failure Margin of 3: \textit{``Look, I told you I don't want to go to the Bearded Mountains any more.  If the king abandoned the place, why do you think it's good enough for us anyway?  I need to get to Quennome pronto and that's that - okay, boy?''}.

If they succeed\ldots but barely: \textit{``Sorry about this, but I really thought Town Master Melissa was in the Bearded Mountains and I've only just realised my mistake.  If we could arrange for a different portal I can get you a note of apology from her.  And some silver for your trouble.  In fact here it is up front.  Three silver pennies -- how does that sound?''}.

If they roll a success Margin of 3: \textit{``Ye gods, man!  You mean Melissa isn't in the Bearded Mountains region?  She'll be livid when all the information about \dots no, NO! I'm afraid I can't say.  King's business and all that.  Military secret.  I'm afraid I'll have to change route to Quennome then -- let's make it snappy, man!  Of course I understand you might be facing trouble for taking people to a different portal.  Look, here's some pennies to get yourself a new ear-hole after the sergeant has a go at you.  Just tell him we had important business, there's a good chap''}.

This method of players rolling before roleplaying to indicate their roll gives value to the social characters' Traits and legitimacy to the antics of more socially clumsy players saying all the wrong things.  The roll of the dice also acts as a way of saying `I am about to speak', so people can pace conversation without interruption.

\subsection{Damage, Death \& Dismemberment}

Losing HP is a massive, screaming deal in First Blood.  It's easy to take habits over from other games where losing one's liver is all part of a normal Tuesday afternoon but here PCs should lose Fate Points, then attempt to flee and only in the most dire situations should they start to bleed.  Damage which doesn't hit home can be brushed over with a brief note about `avoiding the swing' but if anyone loses a single Hit Point the GM should grind the description and combat to a halt to emphasise exactly how eyeball poppingly, knee-cap shatteringly painful and side-splittingly debilitating a knife can be.  Take your time.  Make the words secrete congealed blood.  If the PCs start to lose HP and don't realise how serious this situation is they might perish where they otherwise would have run away to fight another day.

If a PC dies, the player should be slotted into the adventure at the next available opportunity with a new character. If there is no plausible way to insert another character any time soon, consider providing an NPC for them to play - it doesn't need to be one with amazing Traits, just someone who can speak and interact with the world.  All unspent XP from the old character should be given to the new one, allowing the player to boost the NPC or save up more for when they finally make their own character.

If the PC is wounded to the point of being useless then that player is not going to have a lot of fun with the character. If possible, the player should be given a new character for that one adventure, but all XP gained can be kept (if the player wishes) and given to the old character at the end of the adventure. To put it another way, players, rather than characters, hold XP values which can then be placed on any character.

\subsection{Advanced Play}

At a certain point of player advancement, it will be necessary to change the state of play in order to hurt or even threaten the PCs.  If a player gains a Dexterity Bonus of 3 and a kite shield, their TN would be 13, so any creature with a 2 to Strike would have an adjusted TN of 11 to hit them -- a long shot indeed.  This might seem a hopeless state of affairs, so the GM must be careful to employ good tactics.

Firstly, remember to up the basic stats.  There is no problem in simply admitting that well-trained warriors in shining full plate armour will not be brought down by a band of eight goblins wielding sticks -- it's to be expected really.  You have some big, bad creatures in Fenestra.  Use them.  Have the players encounter a dragon.  Perhaps they interrupt two sleeping basilisks and can't escape except by wading through a swamp.

Of course there is still room to use the little creatures with powerful PCs, but you have to use them well.  Never mind an encounter with six goblins -- try an encounter with twenty.  Three surround each player, four of them flank the player to reduce their Evasion Factor by 6 and the last one just attacks, hoping to get a Vitals Shot.  Add a wizard who conjures a swarm of bats to distract them and further reduce their Evasion Factor.  If players are using advanced tactics, you must employ your own.

You can also create your own adventuring parties on opposing sides.  Try one spell-caster who boosts their team's Fate Points before the battle, alongside five warriors with high Attributes and good Combat Skill ratings.  Deck them each out in Complete plate armour (including their spell caster).  The spell caster might also have enchanted an ogre with the enchantment sphere, or a couple  of griffins using the aldaron sphere.  Add in a powerful magic item, but not one which the PCs can use after the battle.

Advanced adventuring parties can also make themselves extremely difficult to hit through shields and focussing their Combat Skill on increasing their Evasion Factor.  The players will have a hard time hitting them without pushing resources into attack rather than defence, at which point they open themselves up to attack.

Remember that the scenery too can be a powerful foe.  Give the players a chilling storm and a volcano to run away from.  Once they accrue Fatigue from running all day, hit them with a night-time encounter with something nasty and give them penalties for fighting uphill on shaky ground.  Add a raging river to cross.

In all cases where you set the players against horrific situations, remember to make the entire thing feel grand.  If the players are fighting against a literal army, they should feel that while everything is bleak, their very survival is an amazing occurrence.

\section{Mana Lakes}\label{mana_lake}\index{Mana Lakes}

Throughout the world, there are little nodes where mana wells up from the ground like a wellspring.  In these locations, some number of MP is created each round which then raises the current MP of mages or nearby magical items.  Whoever has the most empty mana slots - whether an item or a person - gets that mana point.  It pours into the biggest vacuum like water into a hole.  So if a gnome with a current maximum of 6 MP spent 4 MP, that would mean they had a deficit of 4.  Another mage with a maximum mana score of 3 could only get a point after that gnome.  If two are tied for having the biggest deficit, ties are broken by the win going to the highest current maximum MP holder, then the highest Wits Attribute holder, then the highest Intelligence Bonus holder.

Mana lakes also extend the range of all spells cast within to the entire mana lake at least as spells travel easily from one end to the other.  Each has a level, denoting the number of MP it doles out each round, so a mana lake of level 3 would give out 3 MP per round.

Spell casters who can create magical items are few and far between -- perhaps a dozen might exist at any one point in Fenestra.  However, the items they create remain long after death.  Many delight in placing their magical items in mana lakes where the items can remain eternally charged with magical power and ever-ready to do whatever it was they were built to do.  A selection of such mana lakes with magical items (or some other permanent magical effect) are given below.  If your PCs encounter a mana lake on the Random Encounter tables, or you want one for one of your own stories, just pop one of the following into the map.

\subsubsection{King's Hand Valley}

\textit{Level: 2}

There is a valley through which raiders used to come.  To put a stop to it, a priest of Qualm\"{e} cursed it with a mummified hand from one of the kings of long ago.  Nobody has any idea how he obtained the hand.

If, at any time, a dead humanoid body is in the area, it rises from the dead to eat the souls of the living.  Most of the time such dead creatures like to stay and feast off the ambient magics in the area.  They hunker down under snow, or red, squelching leaves, or simply lie and let the moss grow over them.   At first impatient traders would attempt to sneak across the valley as a short cut.  Later, only a few brave warriors would come to test their strength.  Now none come, and the dead are being eaten away by the moss.  But they will not leave.  They simply remain, and wait to hear human speech again, and feel human warmth.  It has been a long time since their last meal and they have gathered a decade's worth of hunger.

The hand hangs from a tree, and looks very much like a hanging vine or a part of the tree it hangs from.  It holds a total of 7 MP and spends 2 to raise anyone in the area as a ghoul as per the second level of the necromancy sphere -- it cannot be activated at will but automatically activates for as long as it has MP to use.

\subsubsection{The Petrified Forest}

\textit{Level: 1}

One day during the season of C\'{a}lea, a great flood came and washed a marsh clean out.  Mud trickled out for the entire season and when all was done, a petrified forest was uncovered.  Stone cylinders from one to six feet tall were uncovered throughout this mile long valley.  The valley is currently covered in grass and fungi but is otherwise uninhabited by any vegetation.

Around the centre of the valley, anyone mentioning something will soon see an illusion of that thing.  The illusions each have a total of 6 points to divide between Speed and Dexterity Bonuses, but have no other Traits.  People moving through might imagine what the forest was once like might see the petrified trees turning into real trees.  Those worrying about monsters will see an illusion of a monster.

The illusion holds 15 MP in total and can spend 3 at any point in order to create an illusion of up to 5 squares size.  Basic illusions are TN 11 to identify as illusions while larger illusions are only TN 9.

\subsubsection{The Myriad Web Forest}

\textit{Level: 1}

In the middle of a great forest sits a massive and majestic shiva tree.  A guardian spell was placed on it long ago by a priestess of Laique such that if anyone enters the area, marauding animals are summoned to the area. This is no problem to an initiate of laique, as they can typically charm such animals -- but anyone who was not a friend of the forest could be in serious trouble.

Since that time, a massive nura spider has taken up residence in the area.  This spider, grown to gargantuan size and twisted by strange magics, has made its home here.  It perpetually traps summoned animals in its web then eats them.  The summoned animals do not leave until they have been killed or leave the area.  However, the nura spider does not simply kill them but bind them with its web and slowly liquefies them with its poison.  This blurring of the boundaries of when the animal is dead or alive, of where the web ends and the creature begins, means that the animals so summoned remain after the spell has ended.

At any given time, its web is full of up to a half dozen assorted creatures -- griffins, aurochs, a bear and such.  The web must be laid slowly, but once on the ground or stretched between branches and tree trunks it is extremely strong.  It counts as having a Strength Bonus of 4 and loses 2 Strength each day as it continuously degrades.  Anyone caught with a lower Strength rating simply cannot break free.  Those with an equal Strength Bonus can make a Dexterity plus Larceny check, TN 8, to break free.

During her time here, the spider has become incredibly large and fat and has given birth to many children.  Some move away to terrorise nearby farms.  Others remain and usually starve to death after failing to fight her for food.  She has one mate -- one male spider with whom she mates.  When hunger overcomes her she sometimes eats another section of his legs.  Currently, he has two and a half legs left.  He cannot move properly as she she has bound him in webbing and left him on branch near the top of the shiva tree, hanging like her prey.  Every so often she approaches the top to mate with him, or to stop one of her children attempting to eat him.

The shiva tree holds 11 MP in total and can conjure animals as per the second level of the conjuration sphere for 2 MP.  It does so and then the mana stays spent for as long as the animal is alive and in the area.  Once one dies, the spell regenerates its MP and with a wave of mist another animal appears ready to fight off attackers.

\subsubsection{The Wishing Wellspring}

\textit{Level: 1}

Long ago gnomes who lived near this area thought it would be fun to create a functioning wishing well for anyone who needed it.  They found a natural wellspring deep underground and created a layer of polished stone on top, complete with a gazebo.  The gazebo's roof has since deteriorated but the stones are still smooth.  Many have minor gemstones set into them, creating a sparkling effect all along the rocky floor when the sun sets.

In the centre of this stone floor placed in the middle of a large plain is the well.  The bucket and ropes were reinforced with steel wiring and function to this day, though they could do with being replaced.

Any time someone makes an audible wish, puts the well's bucket down and drops a coin into the well, a summoning spell activates and creates the thing they wished for inside the bucket.  When it is drawn back up, the wisher can grab the item.  The well holds a total of 16 MP and spends 3 MP in order to create items.  If the well spends mana to summon an item, the mana must, of course, remain spent, so the well will be depleted by 3 MP for as long as the item remains in use.  If the well runs low on mana because too many people are making wishes, it simply cancels the oldest spell it has cast and the wished-for item disappears, whether it is by the well or a hundred leagues away.

Some time after the well's creation, a priest of Qualm\"{e} came and noted how some people were disrespecting the well and were using it too commonly.  Their wishes threatened to break the magic which allowed the well to function.  He decided to protect the well with a spell of his own.  He carved a message in elvish -- the tongue of all academics -- into the side of the well.  It proclaims ``Mine m'{e}re ilyain er'' - ``Only one wish for everyone''.  The skull of an older priest of Qualm\"{e} was placed at the bottom of the well to guard against misuse.  If anyone attempts a second wish a storm of lightning and fire erupts from the well and spreads across a 5 square diameter, dealing $2D6+2$ Damage to everyone there.

Illiterate farmers from the area have been killed by the well, and their skeletons lie there still, what little has not yet been blown away by the stormy seasons.  They regularly warn travellers not to enter the area.

\subsubsection{The Dwarvish Prophet}

\textit{Level: 3}

Deep underground there are many natural streams, flowing quickly into the depths.  Some of these run alongside passages which the dwarves carve out between the various realms they inhabit or realms they inhabited.  It is not uncommon for underground travellers to wander along natural caverns only to later find a dwarvish road to take them part of the journey.  On one of these roads is a carving of the face of a great dwarvish prophet.

The massive face has no beard, leading almost everyone to believe she was female.  A waterfall flows down her face and has worn down everything but the nose, leaving a single, perfect nose and part of the lips while the rest of the face looks more and more like bare rock with each passing decade.  There are various possibilities as to who the face might represent -- there have been many dwarvish prophets and almost half were female, but the exact identity has never been discovered.

As people pass through, runes around her head glow and either curse or bless the target.  Exactly which happens is not well understood.  Some say that spells themselves can become insane after long centuries sitting along in the dark.  Understanding whatever strange formula may exist for who gets blessed and who cursed is especially difficult as most people do not know if they have been cursed or blessed, they merely infer from how well their luck goes later.

Those the statue-head deems worthy receive $2D6$ FP (costing 2 MP).  Those they deem unworthy lose $2D6$ FP.  The statue head holds 10 MP in total.

\subsubsection{The Resting Grove}

\textit{Level: 2}

This group of trees and bushes looks like a lucky find.  There are apples, blueberries, raspberries and tiny strawberries growing all around.  Even better, when one piece of fruit it picked, more grows in its place.  If it rains, the trees naturally knit their branches together  to form an overhead canopy thick enough to keep the rain out.  If people nestle up to their trunks, they can warp and open up, allowing people to sleep inside the newly created hollow.  The entire grove is a luxurious place which caters for the every need of anyone within, including creating simple items like tables or goblets out of wood.

Once people leave the area, any items they have remain as they were, but the trees return to normal.  Any items stored in those temporary shelters is absorbed by the tree, and further mutating magics will not necessarily bring the items back because the tree might not open out in the same way the second time it opens.

The grove contains 13 MP in total and spends 4 in order to make any alteration or 1 MP to calm any wild animal in the area.  All spells cast count as having an Intelligence Bonus of 3.

While this place was once a welcome resting stop for many a traveller, or at least for those brave enough to go near such strange magics, it has since been taken over by bandits.  They use the trees as a continuous supply of arrows for their bows.  They sit in treehouses at the top and shoot down to any attempting to get them out of the area.  Since the grove is not far from a major road, the local town master has made a high price on the bandits' heads.  None have claimed the bounty yet.

\section{Magical Items}\index{Magical Items}

\subsection{Alchemical Items}

\subsubsection{Bag of Tricks}
This gem-studded bag has a tough, silver chord around the top, and won't open more than the width of a man's wrist.  By simply naming an object, people can pull out anything they desire.  Unfortunately, the bag only speaks the gnomish language.

It stores 4 MP, and spends one produce any small, valueless item or substance.

\subsubsection{Bag of Holding}
This looks identical to the bag of tricks, but instead of summoning items, it leads to a closet.  The bag is in fact a magical portal, held permanently open, and allows the owner to deposit and remove items at will.

Exactly how much space is available depends upon how long the user's arm is.  Inside is completely dark, and the folds of the bag (or perhaps further magic) do not allow anyone to peer into the bag -- only to fondle the various shelves resting inside.

Anyone putting an item into the bag in a hurry must make a Dexterity plus Crafts check at TN 6 or risk dropping the item instead of placing it on one of the dark shelves within.  Alchemists have theorized about whether the bag leads to a bottomless pit in some demonic realm, or simply a larder.  Either way, nobody's arm's long enough to reach the bottom.

\subsubsection{The Crystal Dragon}
During the reign of Rex Hunter, a dragon came to Mt Arthur, ate its fill of farmers' livestock, and then ate any humans in its way.  An alchemist of the time was paid handsomely to teleport the thing to a faraway land.  He delicately chiselled out a magical item which would perform that single feat.  

It holds 5 MP at maximum, but currently has 0.  If anyone breaks it or attempts to draw mana from it, it's generally believed that the dragon will return.  Others say this is nonsense, and that once the spell has been released it will be free to banish others to the same distant land, or be used as a powerful mana stone.

\subsection{Divine Items}

\subsubsection{Arthur the Sleeping Cat}
A priest of Laique once left all her worldly mana to a cat, imbueing it with a single spell.  The cat can instinctively make animals of any kind fall asleep.

He holds a total of 16 MP, and spends 3 to activate the spell.  Victims can resist with Wits plus Vigilance, TN 11.

\subsubsection{Deck of Cards}\label{deckofmanythings}
This deck, presumably fashioned by various priests of Alass\"{e}, holds a number of cards.  Once grasbed between \emph{three} or more fingers, the card activates.  If it's held, the target is the caster.  If thrown, the target is whoever the caster throws it at (the caster doesn't need to be a good thrower for the magic to work).

Each card looks much like a normal playing card, with one non-obvious flaw, such as a joker who's wet himself, or a king who's lost his crown.  The spells are each cast with an effective Intelligence and Wits bonus of +4.  Only a few cards remain.  Once one is pulled, the effects activate immediately and lasts for a full scene.

The Deck of Cards resists all magic, including gaining information about it through Fate.

\begin{tcolorbox}
	\begin{enumerate}
		\item{Illusion of an Ogre appears}
		\item{Nearby animals gather to the caster -- roll an encounter}
		\item{Nearest Target becomes insisible}
		\item{Nearest Target turns into a chicken}
		\item{Mist fills the area}
		\item{Nearest Target turns into a bear}
		\item{Blinding light}
		\item{Darkness fill the entire area}
		\item{Nearest Target turns into mist}
		\item{Illusion of a tree growing}
		\item{Nearest Target turns into a gnome}
		\item{Illusion of a portal to the Realm of Darkness and Fire}
		\item{Nearest Target sees a vision of the next encounter}
		\item{All nearby vegetation begins growing}
		\item{Nearest recently deceased target gets to reroll a Vitality Check at +5.  If none, nearest target gains $2D6+4$ FP and can store an additional 10 FP}
	\end{enumerate}
\end{tcolorbox}

\subsubsection{The Raven Tree}
Anyone taking an apple from this tree instantly transforms into a bird of some kind, unless they are strong enough to overcome the spell.  The magic can rearrange up to 7 Attribute points in order to reduce their Strength to -5, and can add up to 7 points to their Speed to bring them high enough to fly.  Once the spell has completed, the target remains as a bird until sundown.

The tree holds 22 MP, and regenerates 3 per scene.  Apples taken from the tree keep their magic for up to a week if kept in good condition.

\subsubsection{Ring of Asphyxiation}

This simple ring, inlaid with bone, act as per the first level of the Necromancy sphere to stop the user feeling the effects of cold or hunger while the ring is worn.

\subsubsection{Venomous Seeds}

This bag of seeds forms a one-use item.  Once opened the bag produces the effects of Wind Blast, moving everyone around back by 4 squares.  Effected creatures lose 4 initiative.

Once the seeds have laid, if they find any kind of purchase -- whether mud, soil or marsh -- they instantly grow into thorny plans, full of venom.  Anyone touching them sickens, gaining 6 Fatigue Points.

\subsection{Runic Items}

\subsubsection{The Eternal Warrior's Armour}

This plate armour, inscribed with necromantic runes, raises the wearer from the dead as a ghoul.  They're typically used by dwarvish suicide squads and worn only after making a solem pledge that the wearer will never stop fighting, even past death.

The breastplate contains two spells -- a second level spell to ensure the wearer rises from the dead, and a first level spell, ensuring the wearer never attacks another dwarf.

\subsection{Songs}

\subsubsection{The Chocking Song}
The song of growth is sung by village elders in the Bearded Mountains regions.  The elders sing only on their own birthday, thanking Laique for their village's prosperity.

The song is special as it sounds most pleasing when sung by someone devoted to Laique, or a follower of the Code of Experience.  Conversely, those following the Code of Acquisition, or who follow Qualm\"{e} or Ohta, reliably choke and cough while attempting the song.

The song holds a total of 17 MP, and needs spend only 1 when sung.

\subsubsection{The Goodnight Sonnet}
This powerful enchantment puts to rest the nearest 14 people to hear it.  They must make a Wits plus Vigilance roll at TN 12 or fall into a deep sleep.  Alternatively, targets can spend 5 FP to ignore the spell.

The sonnet requires a mandolin and voice to be used in harmony, so not many can manage it.  The piece requires a few minutes to play before taking effect.


