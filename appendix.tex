% !TeX root = ../main.tex

\renewcommand{\headingtype}{APPENDIX}

\chapter{Astronomy}\label{astronomy}\label{seasons}\index{Astronomy}
\setcounter{list}{0}\setcounter{enc}{0}

\begin{multicols}{2}

\begin{tcolorbox}
\begin{list}{\addtocounter{enc}{1} \bf Cycle \Roman{enc}}{}

\item :

\begin{list}{\addtocounter{list}{1}\roman{list}}{}

\item \textbf{Qualmea:}  Stormy eclipse

\item \textbf{Atya:}  Mild

\item \textbf{Alassea:}  Cold, eclipse

\item \textbf{Cantea:}  Mild
\end{list}

\item :

\begin{list}{\addtocounter{list}{1}\roman{list}}{}
\item \textbf{C\'{a}lea:}  Warm

\item \textbf{V\'{e}rea:}  Mild

\item \textbf{Otsea:}  Storm

\item \textbf{Toldea:}  Mild

\end{list}

\item :

\begin{list}{\addtocounter{list}{1}\roman{list}}{}

\item \textbf{Laiquea:}  Warm

\item \textbf{Quainea:}  Mild

\item \textbf{Minquesta:}  Cold, eclipse

\item \textbf{Ohta:}  Mild

\end{list}

\end{list}

\end{tcolorbox}

\noindent Circling around the Sun at an amazing speed is the great planet Ainumar.
It is similar to Jupiter, in that it has a constant raging storm across its face and because it is faintly beige.
Ainumar, it is popularly believed, is where the gods live.
Around this massive, roiling bulge in space, with storms racing across its surface faster than any arrow flies, is a green moon.
There, underneath its surface and in little pockets of the dry land on top, there are people.

The worldmoon swings wildly around Ainumar, coming so close in that it could almost kiss the gods, and then hurtles back to sit in empty space, far away from the Sun, before landing for a moment in the coldest patch of all -- the shadow of the gods.
Here, twice every three cycles, the planet grows deathly cold.
For a few hours, everything goes black.
Usually, Ainumar provides dazzling light at night, and sometimes, after, a grand eclipse lasting several hours.

The planet has its own years, but referring to two types of year could get confusing -- the planet really has cycles -- it cycles around the Sun.
In each cycle are four seasons -- the first is called `Qualmea', in deference to the god of death.
It is full of brutal storms and at its height holds an eclipse so long it can often bleed into a two-day night.
After this comes `Atya' -- a milder season before a deep cold season known as Alassea, when people cheer themselves up by playing pranks, telling jokes and drinking or smoking to excess; at night Ainumar becomes a smaller orb in the sky, retreating like the Sun.
Things settle down again for Cantea, the fourth season, though Ainumar is as far away as ever.

That finishes the first cycle -- the second starts with a warm, Summery season called C\'{a}lea; during the night Fenestra spins round to view a dazzling Ainumar.
The world's shadow races across the face of the home of the gods, creating a dark eye which looks up at the night-time planet.
All the ice in the planet recedes to nothing before slowly reforming again.
The world trundles on from there until at the end of the eighth season the second cycle finishes.
 The last cycle ends in Ohta -- the twelfth season, named after the goddess of war.

\end{multicols}

\chapter{History}\index{History}

\section{A Creation Tale}\index{Creation Tale}

\begin{multicols}{2}

\begin{exampletext}

\noindent Deep beneath the earth, worms the size of many castles roam, occasionally rising to push up the earth into great mountain-like mole-hills.
These worms were created by the gods.

Making such big creatures was difficult, so they all had to work together.
However, the creature proved too complex, so most of the parts were removed before the gods woke it up.
At first, it had too many eyes, so the eyes were removed, but the gods did not notice that they were already a little bit awake, so they walked away, and mated with each other, and eventually those eyes turned into elves.
The worm's face looked like a dragon, so of course it had a beard, but the gods decided this was unnecessary and cut it off.
Where the beard fell, a little dwarf was formed, and quickly escaped by burrowing underground.
When the creature was seen to be too smart, they removed some of its heart to make it less able to think (because our thoughts and souls reside in our hearts).
The heart burrowed underground to become warmer, and turned into a little gnome.
Finally, they decided that its arms were only holding it back in the narrow tunnels it made, so they took its muscular arms off and threw them away.
They scuttled off and later became strong humans.

The gods later noticed all the creatures running around the world.
They wanted one to rule the world, governing it properly, and put fire in their blood so that they could command every part of the world.
At first the job was given to the elves, but they did nothing but sit about in their forests reading poetry.
The gods declared that they would no longer rule and at that moment all the elves in the world swore never to recognise them again.
Later, the gods requests that the dwarves would rule the world, and gave them the gift of runic magic to help rule.
But the dwarves just mined and mined, ignoring everything which was happening on the surface, favouring exploration of the great worm tunnels.
Next the gods gave dominion to the gnomes, and to help them govern wisely told them many secrets about how the universe works and how to bend it to their will.
The gnomes tried in earnest to figure out how best to govern, and the discussion grew and grew but they could not reach a conclusion.
While the gnomes continue to argue to this day, the gods did not have all day, so they decided it was time to speak directly.
They selected humans to commune with, speaking with them directly through divine prayer.

And that, little one, is why we are to rule the earth.
And that is why we can live here, instead of them, and why your dad needs to go and get more land for all the little brothers and sisters you might have later; because if we want to govern the land we have to take it first.

\end{exampletext}

\end{multicols}

\section{The Rise of Rex Dalius Quennome}

\begin{multicols}{2}
\label{h_dalius}\index{Dalius Quennome}

\noindent The following letter was found in the grave of a women who lived with the elves until her death.
It was taken to \gls{college} as an item of historical interest.

\begin{quotation}

	Dearest Clara, I write to you in haste to instruct you to hide yourself.
	Land Master Dalius of Quennome is mad.
	He he entreated the many elves there to aid him on his mad mission.
	We have upset them too much -- our incursions were only to strike at Dalius, but we have rattled the hornets nest and now the elves mean to wipe us out through assassination and I am sure that Dalius is at the heart of all of this.
	During this season of Qualmea, the elves have so far murdered three town masters and a number of village masters.
	Four more gone missing, and a further three went mad and ran away to join the elves under some enchantment.
	They have mostly been caught and put into a prison, but with the leadership in such disarray, many have joined forces with Dalius of Quennome.
	At this rate we shall all be ruled by elvish masters before long, with Dalius as their pet pretending to lead the show.

	As Land Master of Mt Arthur, you are in grave danger.
	Please make haste to the Bearded Mountains before the elvish assassins find, then kill or capture, you too.
	Hopefully now that Qualmea is almost ended, the elves will trouble us no longer.

	Yours forever,

	Your faithful servant, Ghent.

\end{quotation}

From the history books of the Quennome region, written by Runcible Shaw, commissioned by Dalius IV of Fenestra.

\begin{quotation}

	After Dalius struck a deal with the elves, they left him to his own devices.
	It was mostly alone that he managed to conquer the Mt Arthur region, and shortly afterwards conquer Whiteland.
	With half of the area North of the Kingsway mountains under his control, Dalius declared himself `Dalius Rex' -- ruler of all of the North.
	The nobles of each region soon began to pay his taxes rather than fight.
	He used his increasing wealth to purchase more suits of armour from the dwarves of first the Mt Arthur region and later from the dwarves of the Bearded Mountains.
	Before long, he ruled over all the lands of the North.

	Dalius II, was a genius by all accounts, and created the first cash economy.
	He bought coinage from the dwarves and later commissioned additional coinage with his face upon the coin, to show that he would guarantee the money be paid in food if ever the coin was handed in.

	Unfortunately, Dalius II's reign was not easy.
	Earthquakes came, larger and harsher than any which usually come during the stormy seasons.
	They came during Qualmea, at the height of the eclipse, and most of the kingdom was levelled by the earthquakes.

	Dalius II was as good at making friends as his father, and quickly commissioned the gnomes to create portals using conjuration magics.
	With these portals, stone could be transported from place to place quickly and conveniently.
	The portals were also dotted around the kingdom to increase trade.
	Suddenly, people deep in Quennome could purchase fish from the Bearded Mountains.
	Elves would use the portals to visit the Pebbles Islands.
	Dwarves would come from deep underground to trade their gold, their armour and return with mutton and leather from above.
	Some trade even began with the barbaric peoples of the South.

	This was a time of great expansion, and before two generations had passed, every town had a wall and every noble family had a castle once more.

\end{quotation}

\subsection{The Reign of Rex Hunter}

\index{Rex Hunter}\index{The War of Portals}\label{r_hunter}

From the notes of Hokl of Whiteland, gnomish chronicler and Master of Forces.

\begin{quotation}

	We told the firyar not to play with fire.
	We told them to build their castles, be grateful for the help and then leave off.
	We suffered from the grand earthquake as well as any and remembered it for longer than any of them.  

	They asked us to show them how it was done -- how we opened portals from one place to the next.
	We warned them that a gnome may spend many decades studying such materials and still fail to produce results, yet still they insisted on pushing their young ones to the point of breaking through arduous study so that they could eventually learn our tricks.
	They made a building, ostensibly dedicated to learning of all kinds, but in practice focussed almost exclusively to studying the Conjuration sphere so that they could have more portals.
	It was built next to the only gnomish portal which still stood, and bared their engravings.
	It stands there to this day, in Eastlake -- taller than any building which has gone before.
	They managed to hold a lot of it up with force magic.
	Some say that if the bottom half falls in an earthquake, the top half will stay where it is.

	We have all heard the stories from the dwarves about what happens deep below the earth -- the strange and terrible creatures which live there, always hungry, always eating.
	We knew well enough not to touch them.
	But humans cannot understand something in the abstract.
	You tell them that a horse kicks and they have to smack its arse before they believe you.

	The humans travelled far and wide, and opened up portals to strange other worlds.
	Worlds of the dead.
	Worlds where secret things happen which nobody was meant to know about until the right time.
	Worlds of deep, dark, hungry beasts.
	We called them `nura', `the deep ones'.
	The humans opened up portals all over the kingdom, sent in explorers and often had to shut the portals off before anyone could get out again after some strange beasts exited and decided to make a meal of this new land.
	For a time it was chaos, but the chaos was at least largely restricted to Whiteland.

	Then the device came.
	Some say it was invented by a human.
	Some of the humans say it was invented by a gnome out of vengeance for abusing their art.
	I believe this is wrong but will not pass moral judgement if it is right.
	The device, any time it was used, opened a portal from where it was to the nura world.
	Hordes of creatures which were nothing but mouth and arms came through and ate and ate all they could, and then dragged the humans back through the portals.
	They came back changed.
	The portal could not be closed except by opening another one -- this was the intentional design of the device.

	The device might have been tossed on an island and forgotten about, but humans' first question in regards to every item is how to use it to kill, so they began to make war.
	Troops journeyed South across the Kingsway Mountains, including Rex Hunter himself (who gave the mountains that name at the time).
	They made war, and before armies could be rallied on the other side, they left a man with the device to open all the portals he could, one after another, while marching along the mountainside and fleeing before any creature could attack him.

	As portals opened to the nura land, people were pulled back in, kicking and screaming, then ejected out as horrifying giants, three or four times the size of a gnome.
	They would exit and march relentlessly, fall upon a village, eat all the food within (including half of the inhabitants) and then return to the portal, if it was still there, with captives.

	It was theorised that some powerful magic existed in there which forced people to change form.
	When interrogated, it was found to everyone's horror that the ogres which came out could still remember being men.
	They had such hunger that they were forced to persistently raid.
	They were instructed to bring others back by some unknown creatures inside the portal -- some kind of terrible worshippers of dark gods which live under the earth.

	While the nura were ravaging everything in the Southern Kingdom, Rex Hunter continued to wage war, making sure to keep his distance from the exact spot where the nura were active at that time.
	The device was soon captured by people in the South Kingdom, but instead of throwing it away or destroying it, they decided to take it back and continue the war with their new weapon.
	The war went on, and the only creatures to really benefit were the nura as the device was past back and forth like a ball in some sport.

	Even we were not immune.
	Our many colonies in the South Kingdom were often targeted accidentally.
	We did not hide in those days but had many structures above the ground.
	The nura came, and took dozens of us at a time, and returned us a horrific little creatures, no longer with a love of growing and learning, but only for eating.
	They are the creatures now known as `goblins'.

	The armies of the South Kingdom took the device over the mountains, and since the dwarves had made a pact with the King of the North, they attempted to stop the mountain crossing.
	The device was used there, and the armies fled back as raging hordes of nura greeted the dwarves at the entrance to one of their tunnels.
	The dwarves were soon captured en mass and taken into the portal, to return much taller, and stronger, and stupider.

	The two armies battled back and forth, and the device was captured many time by both sides, like some unwholesome football.
	Meanwhile the humans had trained up two alchemists who were able to summon permanent magical portals.
	Rather than use them for trade, they were used to move army regiments in order to better attack the South Kingdom.

	This was known as the `war of portals', and it from an old word for portal which we derive the name of the North Kingdom as it is known today -- `Fenestra'.

\end{quotation}

\subsubsection{The Epic of Logan}

This is literal, rather than lyrical, translation of the last verses of the Epic of Logan.

\begin{verse}

	His hands finally around the device,

	He had fought many months to obtain it.

	Logan remembered his oath to the king,

	He swore to combat all the enemies of the kingdom,

	But the enemy was Fenestra itself.

	Armies of nura swirled about him

	Afraid of the last touch they had taken,

	But they grew hungry for the device

	There Logan knew what had to be done.

	He entered the portal to the other world,

	And opened it from within.

\end{verse}

\subsubsection{A Recount of the First Logan's Meet}

- From the pen of Carnestel, Elvish Bard

\begin{quotation}

	We really only came together to sing his song.
	He was a popular man so when his epic was finally completed we all came together on the first anniversary of Logan's sacrifice.
	Of course once there, we got to talking -- mainly there were elves and bards at that first recounting as the bards were eager to know the song early as they knew it would be popular everywhere in the land.

	Once everyone was talking, we started sharing knowledge of the magic of song -- how to sing a blessing or strum an enchantment upon people.
	Of course teaching rarely comes for free.
	Bards expect to be paid, even by other bards.
	That night was an exception -- a lot of powerful magical techniques were passed along many over the week's feasting and drinking.

	Around the end of the festivities, word came on the wind that a nearby elven settlement had been attacked by nura hordes.
	Despite the device being gone they continued to come from under the mountain, having made their own portal there.
	That was the only time I know of that elves have been dragged underground to become nura.
	We all feared how they might come out -- many were old and had grown incredibly fast and dangerous.
	It was decided among us that the entire camp of bards would go out and help with our new songs of power.

	The battle was short, but it took half our number and a great many of the remaining elves of the area.
	We destroyed their portal through a great song of undoing which was constructed so that it could only be sung by a choir of six men and six elves.
	Then we collected the dead, and began to mourn.

	During the meeting the various bards had wrapped pieces of their own souls into their songs to animate them and give them magical powers.
	They intended to break those songs in time and regain whatever piece of themselves they had placed into their music.
	But they had died, and would never unmake their own songs.
	They were only songs to summon entertaining illusions, or to cast minor blessings, but they would not be forgotten or misused by any of the survivors.

	We decided then that it would be our duty to fight off nura incursions, or at least to keep an eye on them.
	Many cycles can pass without a meeting, but when a meeting is called, everyone comes.
	It is all done through the power of the spell-songs.
	Let me explain.
	A number of people all know the same song.
	Each one sings the song and activates the spell only once per season -- everyone has his designated day so that they do not break the spell by over use.
	If any one of them ever wants to hold a meet-up, the spell-song is sung until it is broken and the magic is all gone.
	When the others attempt to sing the song, they realise one after the other that the magic is gone and that they must go to the meeting on the next cycle at a prearranged date.

	The humans have all since died of course, but we make sure to teach other humans to replace them.
	We gather and sing songs, we eat and tell the Epic of Logan, and we all exchange any information on encroaching nura, bubbling up from underground.

	Now if you will excuse me, tonight is my night, and I must sing to make sure that my spell-song is still intact and that I am not needed.

\end{quotation}

\end{multicols}

\section{The Rise of Nolan Beard}
\label{nolan}

\begin{multicols}{2}

\noindent From the banned Histories of the Beard Family, by Rafferty Quennome.

\begin{quotation}

	Legend has it that Nolan of the Bearded Mountains was the worst dinner guest in history.
	He would spit as he talked and always boasted of either himself, his powerful friends, or how much he resented the kingdom being orchestrated from a castle in Quennome.
	He had some friends among the dwarves of the Bearded Mountains, mainly because they did not trust the elves.

	He spent a good weight in gold bribing secretive bardic organisations to aid him in sending quick signals.
	The bards have some special magic for it which nobody can figure out.
	He also promised \gls{college} that they would receive extensive funding if they would join him.

	Nolan used portals to march his army quickly to the king's castle in Quennome.
	The castle was sacked there and everyone inside killed.
	At the point when the king died, the high priest of Qualm\"{e} was staying within the castle.
	He made a curse upon the king's body, so that anyone who died in the presence of the king would rise up as one of the hungry dead and feed on human souls.
	The spell did not stop drive back the attackers unfortunately, it only succeeded in making that castle a cursed place forever afterwards.
	It lies in ruin and rot even now, its road since destroyed, somewhere in the middle of the great Quennome forest.
	Nolan was crowned before the end of the week.

	Nolan's reign was terrible.
	His first act was to begin the building of a castle.
	The central hub was built in Whiteland and has no door or window, only a narrow hatch to access the top.
	He then commissioned different castle `wings', each in completely different areas in Fenestra.
	Finally, \gls{college} was charged with the task of creating portals from the various `castle wings' to the central hub in Whiteland.
	By the time of this writing he has succeeded, though he is of course long dead.
	It is known as `\Gls{shatteredcastle}', and has pieces and openings all across the land.
	Each piece has one portal which goes to the central hub, so from any castle one can travel, with the king's permission, to any other part.

	The king's next act was to destroy the band of bards who had given him advanced warning of the movements of the Quennome army.
	He managed to kill only a few, as the rest quickly went into hiding.
	To this day, the organisation is illegal and its continued existence is one of the greatest failings of Rex Nolan Beard.

	Finally, Nolan worried about the dwarves.
	They too could set up portals, and had been known to forge alliances between themselves.
	He worried about being surrounded by dwarvish mountains on all sides, so he began a war with them too.

	The war with the dwarves was difficult and protracted.
	They kept blocking tunnels to fend off his attacks, and while they did not like having to make do without being able to trade with humans for their food, they can be stubborn.
	Eventually, however, a peace treaty was signed.
	The dwarves agreed to pay a small tax, to recognise Nolan's crown and to disband all of their portals.
	In return, they would be given cheap foods and enjoy the protection of human armies if the nura ever arose.

\end{quotation}

\index{\glsentrytext{king}}

\subsubsection{The Reign of \gls{king}}

Continued from the banned Histories of the Beard Family.

\begin{quotation}

	Nolan died choking on his own vomit.
	It is said that his own royal guard heard him stop breathing because his snort was usually so grotesquely loud, even as he slept, yet they did nothing out of pure hatred for him.

	Nolan's son, \gls{king}, took hold of the reigns easily.
	He was a more timorous character than his father and would constantly worry out loud about elvish assassins sneaking into the imperial court.
	He designed childish password systems to ensure no shape changers could enter the building without detection, though the passwords were so impractical that most guards just told each other rather than throwing those who could not remember into jail.

	The bards, meanwhile, were still operational and unhappy.
	They started taking gnolls into their ranks and helping them prepare for war.
	They gathered information about all the portal locations they could.
	Gnolls from Dogland and Whiteland went out to war.

	At this time, a war had already started in Dogland between the gnolls and the humans.
	The bards only had to pass information on to Whiteland and secretly take emissaries from one to the other to strengthen ties between tribes.
	The war grew and \gls{king} a fatal flaw in the design of \gls{shatteredcastle}: the portals meant his army could travel anywhere, but it also meant that they could be attacked from all sides at the same time.
	The gnolls pushed in and met in the castle's centre.
	Troops arrived from Eastlake through \gls{shatteredcastle} wing there.

	\Gls{king}, paranoid and short-sighted as ever, could not break with family tradition for a moment and soon after issued another blundering, kingdom-wrecking decree: all magical portals were banned except those officially commissioned by the crown, and all must go to the heart of \gls{shatteredcastle} so that he could control them.
	We could have created a flourishing kingdom with easy transport and excellent trade to every part of the world.
	We could have stopped ten thousand blisters on the feet of all the tradesmen in the realm who wander through monstrous forest Paths to transport goods.
	But \gls{king} could have none of that, so he has sacked gnomish villages and killed alchemists on the suspicion that some new portal is their doing.
	Profit-seeking alchemists, take note! -- flee to a better kingdom, one where your talents can be accepted.

\end{quotation}

\subsubsection{From the Journals of Gregor Dogland}

\index{Gregor Dogland}

\begin{quotation}

	At the time of the tragic passing of Rex Nolan, I was stationed as a mere commander of one hundred men.
	I quickly rose through the ranks to take my place by my father's side during the Dog Wars overseen by His Majesty \gls{king} Fenestra.
	Many battles were planned meticulously -- I oversaw most of them personally.
	I was in charge of pushing back the invasions from Dogland.
	It was there most of the trouble started.
	Pushing gnoll hordes back, time after time, became wearisome, so eventually I happened upon the plan which won the war: we would create a portal in the centre of Dogland and wage perpetual warfare without allowing the gnolls to fall back to safety at any point.
	Naturally, it connected with the Heart of \gls{shatteredcastle}, as all portals must.

	While previously we had been putting captured gnolls to the sword, \emph{I} set them to work.
	Before the war was won we had a flourishing guild all based around the labour of gnolls.
	Their first task was to build another wing of \gls{shatteredcastle} in Dogland.

	As the months went by I built up a healthy guild of gnolls.
	We would take them out, set them to work on someone's castle, or digging some hole, and then charge a very reasonable fee.
	Within a decade we had built up a good civilisation within Dogland, for all manner of families to come and purchase affordable farms.

	Of course, it was not long until trouble started anew as ships from the South came to invade our islands and threaten our shore.
	The War of the Pebbles began before the last gnoll in Dogland was put to use.

	The War is almost over, but the crown has further troubles in Whiteland.
	The nobles there have attempted to erect a magical portal to trade with those in Mt Arthur and refused royal decrees to destroy it.
	His Majesty \gls{king} has dealt with the situation, and Whiteland is to have no more nobility until those people can be trusted to toe the line.
	It shall remain ruled by independent village masters but by nothing higher and a bounty has been placed on the heads of any surviving nobility from Whiteland.

\end{quotation}

\subsubsection{Current Affairs from the Bard Lennox.}

\begin{quotation}

	Peace!

	Ladies and gentlemen, it has been ten long years of warfare at sea with the South Kingdom, but at long last we have peace.
	The gnoll raiders in Whiteland have fallen back to lick their wounds, and the wicked nobles there who attempted to split the kingdom have been vanquished.
	The ships of the South Kingdom have vanished back to their own ports or sunk in the sea.

	At last, peace reigns.

\end{quotation}

\begin{quotation}

	A handful of generations ago Dogland was populated mainly by gnolls.
	A few human villages settled around the borders alongside the tamer gnolls while deeper into Dogland were wilder creatures.
	Because of Dogland's position, none could access the sea through this region, and had to travel from Mt Arthur, up Eastlake and then down to the Bearded Mountains area in order to reach the sea.

	The gnolls close to human inhabitants would trade what they had hunted in return for various human wares -- they detested the milk but loved the bountiful supplies of pork.
	They always appreciated a well-made sword though swords became mostly a status-item among the gnolls as they would still have to hunt with spears.
	Many also took to wearing clothes or even purchasing human armour and learning basic crafting techniques.

	Unfortunately the peace broke when a human trader was caught on his own and the gnolls considered him fair game.
	He was killed and his wares taken.
	The humans said that the gnolls were not smart enough to hide their own crime.
	The gnolls who reported the murder stated that they had nothing to do with it as this was the doing of the Longtongue tribe rather than their own.
	Unfortunately few of the humans in the area recognised the distinctions between tribes and war soon broke out.

	The humans of the area could not hope to combat the gnolls on their own -- gnolls are fast and strong, and experts in guerilla warfare.
	However, help was given from the Mt Arthur region.
	Eventually the war was won by Gregg Mt Arthur.
	The gnolls were not simply killed, but enslaved and set to work cutting down trees, building houses and sometimes porting stone.
	A magical portal was placed inside a castle to transport stone from Dogland's mountains and quarries to places all around Fenestra.

	In only two decades, entire villages were built alongside a few castles and other fortifications.
	The gnoll population was bread and exploded -- where hunting could only sustain a few hundred gnolls, human farming techniques meant greater food production, which allowed thousands of gnolls to fill the area, all in chains.

	Nowadays, Dogland is the seat of the Gnolls Guild -- a guild which builds entire villages or even castles with the aid of massive numbers of labourers.
	The villages there are new and populated almost entirely by people who came from the other regions within the past two generations.
	They have a booming sea trade as new ports are opening all along the sea-front to trade with the Bearded Mountains region and the Pebbles Islands.

\end{quotation}

\end{multicols}

\chapter{The Night Guard}
\label{nightguard}

\begin{multicols}{2}

\noindent
The Night Guard hosts humanity's front-line protectors, but also beckons the poor towards the jaws of all the beasts of Fenestra.

The Night Guard typically dress all in black or dark greens.
The darkness helps them lay ambushes for monsters.

The Night Guard recognise a strict hierarchy, in a very particular order.
Whoever the highest ranking person around is has the ability to override others.
The following stations are presented in order, so Scouts outrank Associates, and Associates outrank Soldiers.

\subsection{Novices}

New recruits are typically sent to Whiteplains to train for at least six months.
The journey is long and tiring, and the rations are so poor that many come back with less muscle than when they first arrived.

Novices sleep and eat in the barracks, and receive 20cp per week -- practically nothing.
About 30\% die during their first mission, so they tend to drink all of their wages to cope with the stress.

Novices are given basic leather armour and a short sword, and are expected to keep them in good condition.

\subsubsection{Duties}

\begin{itemize}

	\item{Guarding livestock overnight}
	\item{Keeping watch around the town wall}
	\item{Patrolling a town}
\end{itemize}

\subsection{Soldier}

Once a novice has felled their first monster with a close-quarter weapon, they are officially soldiers.
A lot of leeway is given to these soldiers, and often Lieutenants will allow someone to raise rank by killing a bandit, or by simply getting a good hit against some monster which was killed by the team.
Despite the leeway, arguments about who did what are common among new recruits.

Full members of the Guard can expect a full 100cp per week in payment, as well as the ability to sleep and eat in the soldiers' barracks.
Townsfolk and villagers alike typically respect them because of the difficult duties they perform.

Soldiers can request replacement arms at any point, but will typically have to accept whatever weapons are available -- generally a short sword and leather armour.

\subsubsection{Duties}

\begin{itemize}

	\item{Rounding up a village militia to fend off a basilisk}
	\item{Fighting monsters}
	\item{Tracking down thieves in town}
\end{itemize}

\subsection{Associate}

Associate are those who join the Guard for a particular mission, or just to advise.
Their pay varies greatly, but they are given the title and called `Associate Oscar', or `Associate Maria', in order to give them a fixed position.
Associates can tell Scouts what to scout for, but are expected to do as they are told by Lieutenants.

Having `Associate' as a formal and prestigious title also helps keep undesirables out of the Night Guard.
It might be tempting to hire some local thugs for a short mission, instead of paying people a long-term salary, but the rules on the matter are quite clear -- those thugs would outrank normal soldiers.
This potentially horrifying situation deters almost all Lieutenants from hiring anyone who is not minimally competent.

\subsubsection{Duties}

\begin{itemize}

	\item{Killing rogue magic users}
	\item{Hunting massive monsters}
	\item{Reporting on village masters who threaten the Rex}
\end{itemize}

\subsection{Scouts}

Those Night Guard with a special talent for sneaking and surviving in the wilderness become Scouts -- the elite troop who stay away from the base for long periods of time.

They earn 1gp per week in recognition of the additional dangers they face, and in order to pay for the expenses of travelling, such as paying for their own food.

Scouts who are caught staying with villagers when they were not called for can expect a demotion before long.

\subsubsection{Duties}

\begin{itemize}

	\item{Tracking down monster nests}
	\item{Uncovering illegal weapon stockpiles}
	\item{Spying on other members of the Night Guard and reporting the local Captain}

\end{itemize}

\subsection{Masters}

Village masters, town masters, or anyone else with a noble title may command the Night Guard to protect, serve, or do anything else they wish, so long as no Lieutenants have given contrary orders.
Such masters are not \textit{in} the Guard in any way, but the soldiers are theoretically there as a replacement to the guards that such nobles would organize on their own, before the time of Nolan Beard.\footnote{See page \pageref{nolan} for more on the monarchy.}

\subsection{Lieutenant}

To achieve the rank of Lieutenant requires three things.

\begin{enumerate}

	\item{Literacy (meaning `Academics 1')}
	\item{A recommendation from a Captain to a Prefect or Commander}
	\item{Some outstanding achievement on the battlefield}

\end{enumerate}

\noindent
If all this checks out well, Lieutenants earn 6gp each week.

The job of a Lieutenant is to lead the Night Guard to complete any jobs that others have failed at.
Specifically, if two or more groups fail at some task, the Lieutenant is bound to personally deal with the situation (along with as many men as they care to bring).

\subsubsection{Duties}

\begin{itemize}

	\item{Leading men into battle}
	\item{Recruiting new members}
	\item{Keeping track of soldiers and novices}

\end{itemize}

\subsection{Magus}

Typically, a Magus is a human alchemist from the \gls{college}, however priests of Ohta or Qualm\"e have been known to lend their martial skills to the battlefield.
Force mages use magical shields to protect wounded soldiers and scout ahead with magical senses.
Invocationists destroy enemies with fire.
Conjurers are rarely seen on the battlefield as \gls{king} has ordered anyone showing promise in this field to focus on the study of magical gates.

In theory, a Magus should not go onto the field without extensive training, but in reality \gls{college} often selects the least useful mage they have to go to war, then throw a few snowballs at them to prepare them for real fighting.
A magus is typically allowed free access to buy anything by request, but they are not paid.
\Gls{king} worries that with too much political power, they might become dangerous.

\subsubsection{Duties}

\begin{itemize}

	\item{Strategy and planning}
	\item{Preparing temporary magical items for particularly dangerous missions}
	\item{Occasionally accompanying soldiers on dangerous missions}

\end{itemize}

\subsection{Observer}

The petty bureaucrats from \gls{shatteredcastle} who occasionally come to observe, complete reports, and pester Lieutenants, are a perpetual menace to the Night Guard.
They have no field experience but still outrank even a battle hardened magus.

They receive only 1gp per week.

\subsubsection{Duties}

\begin{itemize}

	\item{Routing out any challenges to the Rex}
	\item{Making sure everyone gets paid on time}
	\item{Advising Lieutenants on tactics}
	\item{Keeping an accurate count of the number of Night Guard}
	\item{Collecting taxes from masters}

\end{itemize}

\subsection{Captain}

Captains organize all the Night Guard around a town.
This might include five to a hundred villages in the area.
Each village makes requests directly to the captain for aid, and the captain then decides how many men to send, and where.

Captains are selected from Lieutenants when an old Captain dies, or when a new town is founded.
Payment is 10gp per week.

\subsubsection{Duties}

\begin{itemize}

	\item{Ensuring the local town and surrounding villages remain free from nura}
	\item{Collecting reports from the scouts}
	\item{Making reports for Prefects}

\end{itemize}

\subsection{Commander}

Commanders reside cities, and spend most of their days book-keeping, and making reports to Prefects.
They are chosen by Prefects, just whenever those prefects desire.
Commanders are never expected to fight, but should have some expertise in strategy in case of war.
Commanders nominally receive 20gp per week -- enough for a modest estate with servants -- but most bolster their pay with bribes.

\subsubsection{Duties}

\begin{itemize}

	\item{Liaising with village and town masters}

\end{itemize}

\subsection{Prefect}

Prefects are the bureaucrats from the \gls{shatteredcastle} who control the Night Guard.
They have no military background, but have the final say on all matters concerning war.
They travel from town to town, or area to area, often with fifty or more soldiers.

The prefects are given leave to bargain with town masters on behalf of \gls{king}, and for the most part have free reign to do as they please.

Prefects live in \gls{shatteredcastle}, typically in the Bearded Mountains or the Pebbles wings.
Each of them show a lot of loyalty to the crown at every opportunity, including disciplining any soldiers suspected of an irreverent attitude towards \gls{king}.

\subsubsection{Duties}

\begin{itemize}

	\item{The peace of the realm}
	\item{Creating the budget for the observers}

\end{itemize}

\end{multicols}

\chapter{Experience Rewards}

\begin{multicols}{2}

\noindent When making antagonists with pen and paper, I'd recommend eyeballing the XP, as enforcing a procedure can turn into a pain.

However, if you're interested in how I've arrived at the XP values here, they are as follows:

Additional points are added for anything which aids these basic Attributes, such as weapons' damage increase, or knacks.

\label{lastpage}The broad idea is that XP should be proportional to how much of a challenge an antagonist presents.
Additionally, some stats work better together than others.
A creature which can reliably hit people but cannot do much damage is less of a threat than a creature which can strike well \emph{and} deal a lot of Damage.
The three attacking stats -- the creature's final \textit{Strike} score, Damage, and Initiative, are multiplied together, as are the defensive stats -- HP, DR, and Evasion.
The result is that if a creature has a lot of HP, adding some more DR will increase the challenge considerably, but adding more Damage will not make the creature much more dangerous if it cannot hit properly.

\end{multicols}

\begin{equation}
\frac{(( Strike + 8 ) \times ( Initiative + 5) \times ( Damage + 5 ) ) + ( ( TN -2 ) \times ( DR + 2 ) \times ( HP )) - 350}{120}
\end{equation}

\chapter{Side Quest Summaries}
\label{sqSummaries}

This complete list of Side Quests allows you to see which Side Quests are available at a glance.
One the PCs encounter some Side Quest which combines with another, pick another Side Quest in the area to continue or start.

\printcontents[Town]{l}{2}{\section*{Town}\setcounter{tocdepth}{3}}

\printcontents[Villages]{l}{2}{\section*{Villages}\setcounter{tocdepth}{3}}

\printcontents[Forest]{l}{2}{\section*{Forest}\setcounter{tocdepth}{3}}
